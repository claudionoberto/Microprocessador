\documentclass{article}
\usepackage{amsmath}
\usepackage{color,pxfonts,fix-cm}
\usepackage{latexsym}
\usepackage[mathletters]{ucs}
\DeclareUnicodeCharacter{8211}{\textendash}
\DeclareUnicodeCharacter{46}{\textperiodcentered}
\DeclareUnicodeCharacter{58}{$\colon$}
\DeclareUnicodeCharacter{32}{$\ $}
\usepackage[T1]{fontenc}
\usepackage[utf8x]{inputenc}
\usepackage{pict2e}
\usepackage{wasysym}
\usepackage[english]{babel}
\usepackage{tikz}
\pagestyle{empty}
\usepackage[margin=0in,paperwidth=596pt,paperheight=842pt]{geometry}
\begin{document}
\definecolor{color_283006}{rgb}{1,1,1}
\definecolor{color_29791}{rgb}{0,0,0}
\begin{tikzpicture}[overlay]
\path(0pt,0pt);
\filldraw[color_283006][nonzero rule]
(-15pt, 10pt) -- (580.5pt, 10pt)
 -- (580.5pt, 10pt)
 -- (580.5pt, -832.25pt)
 -- (580.5pt, -832.25pt)
 -- (-15pt, -832.25pt) -- cycle
;
\end{tikzpicture}
\begin{picture}(-5,0)(2.5,0)
\put(308.5128,-105.7){\fontsize{10}{1}\usefont{T1}{cmr}{m}{n}\selectfont\color{color_29791} }
\put(130.3878,-119.199){\fontsize{10}{1}\usefont{T1}{cmr}{m}{n}\selectfont\color{color_29791}Instituto Federal de Educação, Ciência e Tecnologia de Mato Grosso }
\put(226.7977,-132.6981){\fontsize{10}{1}\usefont{T1}{cmr}{m}{n}\selectfont\color{color_29791}Campus Cuiabá Octayde }
\put(185.947,-146.1971){\fontsize{10}{1}\usefont{T1}{cmr}{m}{n}\selectfont\color{color_29791}Graduação em Engenharia da Computação }
\put(41.7,-160.6341){\fontsize{11}{1}\usefont{T1}{cmr}{m}{n}\selectfont\color{color_29791} }
\put(41.7,-175.283){\fontsize{11}{1}\usefont{T1}{cmr}{m}{n}\selectfont\color{color_29791} }
\put(41.7,-189.9319){\fontsize{11}{1}\usefont{T1}{cmr}{m}{n}\selectfont\color{color_29791} }
\put(151.6523,-208.3948){\fontsize{14}{1}\usefont{T1}{cmr}{b}{n}\selectfont\color{color_29791}TRABALHO DE MICROPROCESSADOR }
\put(167.1988,-227.4935){\fontsize{14}{1}\usefont{T1}{cmr}{b}{n}\selectfont\color{color_29791}CRIANDO MIPS USANDO LOGISIM }
\put(282.6378,-246.5921){\fontsize{14}{1}\usefont{T1}{cmr}{b}{n}\selectfont\color{color_29791} }
\put(282.6378,-265.6907){\fontsize{14}{1}\usefont{T1}{cmr}{b}{n}\selectfont\color{color_29791} }
\put(41.7,-281.9754){\fontsize{11}{1}\usefont{T1}{cmr}{m}{n}\selectfont\color{color_29791} }
\put(41.7,-296.6243){\fontsize{11}{1}\usefont{T1}{cmr}{m}{n}\selectfont\color{color_29791} }
\put(41.7,-311.2732){\fontsize{11}{1}\usefont{T1}{cmr}{m}{n}\selectfont\color{color_29791} }
\put(41.7,-325.9222){\fontsize{11}{1}\usefont{T1}{cmr}{m}{n}\selectfont\color{color_29791} }
\put(225.8902,-344.3851){\fontsize{14}{1}\usefont{T1}{cmr}{b}{n}\selectfont\color{color_29791}R E L A T Ó R I O }
\put(41.7,-360.6697){\fontsize{11}{1}\usefont{T1}{cmr}{m}{n}\selectfont\color{color_29791} }
\put(41.7,-375.3187){\fontsize{11}{1}\usefont{T1}{cmr}{m}{n}\selectfont\color{color_29791} }
\put(41.7,-389.9676){\fontsize{11}{1}\usefont{T1}{cmr}{m}{n}\selectfont\color{color_29791} }
\put(282.6378,-408.4305){\fontsize{14}{1}\usefont{T1}{cmr}{b}{n}\selectfont\color{color_29791} }
\put(41.7,-424.7151){\fontsize{11}{1}\usefont{T1}{cmr}{m}{n}\selectfont\color{color_29791} }
\put(41.7,-439.364){\fontsize{11}{1}\usefont{T1}{cmr}{m}{n}\selectfont\color{color_29791} }
\put(41.7,-454.013){\fontsize{11}{1}\usefont{T1}{cmr}{m}{n}\selectfont\color{color_29791} }
\put(41.7,-468.6619){\fontsize{11}{1}\usefont{T1}{cmr}{m}{n}\selectfont\color{color_29791} }
\put(266.8163,-482.3728){\fontsize{10}{1}\usefont{T1}{cmr}{b}{n}\selectfont\color{color_29791}Aluno: }
\put(154.7555,-495.8719){\fontsize{10}{1}\usefont{T1}{cmr}{m}{it}\selectfont\color{color_29791}Claudio Noberto – claudio.noberto@estudante.ifmt.edu.br }
\put(282.6378,-510.3089){\fontsize{11}{1}\usefont{T1}{cmr}{m}{n}\selectfont\color{color_29791} }
\put(41.7,-524.9578){\fontsize{11}{1}\usefont{T1}{cmr}{m}{n}\selectfont\color{color_29791} }
\put(41.7,-539.6068){\fontsize{11}{1}\usefont{T1}{cmr}{m}{n}\selectfont\color{color_29791} }
\put(41.7,-554.2557){\fontsize{11}{1}\usefont{T1}{cmr}{m}{n}\selectfont\color{color_29791} }
\put(41.7,-568.9045){\fontsize{11}{1}\usefont{T1}{cmr}{m}{n}\selectfont\color{color_29791} }
\put(257.6402,-582.6155){\fontsize{10}{1}\usefont{T1}{cmr}{b}{n}\selectfont\color{color_29791}Professor: }
\put(248.4714,-596.1146){\fontsize{10}{1}\usefont{T1}{cmr}{m}{it}\selectfont\color{color_29791}Ruy de Oliveira }
\put(41.7,-610.5515){\fontsize{11}{1}\usefont{T1}{cmr}{m}{n}\selectfont\color{color_29791} }
\put(41.7,-625.2005){\fontsize{11}{1}\usefont{T1}{cmr}{m}{n}\selectfont\color{color_29791} }
\put(41.7,-639.8494){\fontsize{11}{1}\usefont{T1}{cmr}{m}{n}\selectfont\color{color_29791} }
\put(41.7,-654.4983){\fontsize{11}{1}\usefont{T1}{cmr}{m}{n}\selectfont\color{color_29791} }
\put(41.7,-669.1473){\fontsize{11}{1}\usefont{T1}{cmr}{m}{n}\selectfont\color{color_29791} }
\put(41.7,-683.7962){\fontsize{11}{1}\usefont{T1}{cmr}{m}{n}\selectfont\color{color_29791} }
\put(41.7,-698.4451){\fontsize{11}{1}\usefont{T1}{cmr}{m}{n}\selectfont\color{color_29791} }
\put(41.7,-713.0941){\fontsize{11}{1}\usefont{T1}{cmr}{m}{n}\selectfont\color{color_29791} }
\put(41.7,-727.743){\fontsize{11}{1}\usefont{T1}{cmr}{m}{n}\selectfont\color{color_29791} }
\put(41.7,-742.3918){\fontsize{11}{1}\usefont{T1}{cmr}{m}{n}\selectfont\color{color_29791} }
\put(256.8056,-756.1028){\fontsize{10}{1}\usefont{T1}{cmr}{m}{n}\selectfont\color{color_29791}Cuiabá, MT }
\put(244.8522,-769.6019){\fontsize{10}{1}\usefont{T1}{cmr}{m}{n}\selectfont\color{color_29791}F}
\put(258.2628,-105.7){\includegraphics[width=48.75pt,height=55.5pt]{latexImage_4ae8e5e96d711b41fc007f0a6297c322.png}}
\end{picture}
\newpage
\begin{tikzpicture}[overlay]
\path(0pt,0pt);
\filldraw[color_283006][nonzero rule]
(-15pt, 10pt) -- (580.5pt, 10pt)
 -- (580.5pt, 10pt)
 -- (580.5pt, -832.25pt)
 -- (580.5pt, -832.25pt)
 -- (-15pt, -832.25pt) -- cycle
;
\end{tikzpicture}
\begin{picture}(-5,0)(2.5,0)
\put(113.5464,-59.83179){\fontsize{14}{1}\usefont{T1}{cmr}{b}{n}\selectfont\color{color_29791}Relatório do Trabalho Sobre MIPS Usando Logisim }
\put(188.1734,-78.93042){\fontsize{14}{1}\usefont{T1}{cmr}{b}{n}\selectfont\color{color_29791} Engenharia da Computação }
\put(41.7,-95.21509){\fontsize{11}{1}\usefont{T1}{cmr}{m}{n}\selectfont\color{color_29791} }
\put(41.7,-109.8641){\fontsize{11}{1}\usefont{T1}{cmr}{m}{n}\selectfont\color{color_29791} }
\put(263.6447,-125.451){\fontsize{12}{1}\usefont{T1}{cmr}{b}{n}\selectfont\color{color_29791}Aluno: }
\put(129.1192,-141.2498){\fontsize{12}{1}\usefont{T1}{cmr}{m}{it}\selectfont\color{color_29791}Claudio Noberto – claudio.noberto@estudante.ifmt.edu.br }
\put(41.7,-156.1107){\fontsize{11}{1}\usefont{T1}{cmr}{m}{n}\selectfont\color{color_29791} }
\put(41.7,-170.7596){\fontsize{11}{1}\usefont{T1}{cmr}{m}{n}\selectfont\color{color_29791} }
\put(41.7,-185.4084){\fontsize{11}{1}\usefont{T1}{cmr}{m}{n}\selectfont\color{color_29791} }
\put(41.7,-200.9955){\fontsize{12}{1}\usefont{T1}{cmr}{m}{n}\selectfont\color{color_29791}Disciplina:  Microprocessadores – Normal.3131 }
\put(41.7,-216.7942){\fontsize{12}{1}\usefont{T1}{cmr}{m}{n}\selectfont\color{color_29791}Professor:  Ruy de Oliveira }
\put(41.7,-231.655){\fontsize{11}{1}\usefont{T1}{cmr}{m}{n}\selectfont\color{color_29791} }
\put(41.7,-246.3041){\fontsize{11}{1}\usefont{T1}{cmr}{m}{n}\selectfont\color{color_29791} }
\put(41.7,-260.9529){\fontsize{11}{1}\usefont{T1}{cmr}{m}{n}\selectfont\color{color_29791} }
\put(41.7,-275.6018){\fontsize{11}{1}\usefont{T1}{cmr}{m}{n}\selectfont\color{color_29791} }
\put(41.7,-291.1888){\fontsize{12}{1}\usefont{T1}{cmr}{b}{n}\selectfont\color{color_29791}Resumo: Este relatório apresenta a implementação completa do microprocessador MIPS }
\put(41.7,-304.9877){\fontsize{12}{1}\usefont{T1}{cmr}{m}{n}\selectfont\color{color_29791}de 16 bits, incluindo o controlador principal e o ALUController, que gerenciam o fluxo de }
\put(41.7,-318.7865){\fontsize{12}{1}\usefont{T1}{cmr}{m}{n}\selectfont\color{color_29791}dados e a execução das instruções. O trabalho teve como objetivo compreender e }
\put(41.7,-332.5853){\fontsize{12}{1}\usefont{T1}{cmr}{m}{n}\selectfont\color{color_29791}analisar a interação entre o banco de registradores, a memória RAM e a Unidade Lógica e }
\put(41.7,-346.3842){\fontsize{12}{1}\usefont{T1}{cmr}{m}{n}\selectfont\color{color_29791}Aritmética (ULA), explorando conceitos fundamentais da arquitetura MIPS, como }
\put(41.7,-360.183){\fontsize{12}{1}\usefont{T1}{cmr}{m}{n}\selectfont\color{color_29791}operações de leitura e escrita em memória, decodificação de instruções, controle de }
\put(41.7,-373.9818){\fontsize{12}{1}\usefont{T1}{cmr}{m}{n}\selectfont\color{color_29791}barramentos e geração de sinais de controle. São descritos o funcionamento das }
\put(41.7,-387.7806){\fontsize{12}{1}\usefont{T1}{cmr}{m}{n}\selectfont\color{color_29791}instruções implementadas, a estrutura do banco de registradores, da memória e da ULA, }
\put(41.7,-401.5795){\fontsize{12}{1}\usefont{T1}{cmr}{m}{n}\selectfont\color{color_29791}bem como os testes realizados para validar a comunicação entre os módulos e o correto }
\put(41.7,-415.3783){\fontsize{12}{1}\usefont{T1}{cmr}{m}{n}\selectfont\color{color_29791}funcionamento do controlador e do ALUController no ambiente virtual. A implementação }
\put(41.7,-429.1771){\fontsize{12}{1}\usefont{T1}{cmr}{m}{n}\selectfont\color{color_29791}final demonstra a capacidade do microprocessador em executar instruções de forma }
\put(41.7,-442.976){\fontsize{12}{1}\usefont{T1}{cmr}{m}{n}\selectfont\color{color_29791}eficiente, garantindo a integridade dos dados e o controle preciso das operações. }
\put(41.7,-458.7747){\fontsize{12}{1}\usefont{T1}{cmr}{b}{n}\selectfont\color{color_29791}Palavras-chave: MIPS, Arquitetura de computadores, Microprocessador, Logisim. }
\put(41.7,-473.6355){\fontsize{11}{1}\usefont{T1}{cmr}{m}{n}\selectfont\color{color_29791} }
\put(41.7,-488.2845){\fontsize{11}{1}\usefont{T1}{cmr}{m}{n}\selectfont\color{color_29791} }
\put(41.7,-502.9333){\fontsize{11}{1}\usefont{T1}{cmr}{m}{n}\selectfont\color{color_29791} }
\put(41.7,-517.5823){\fontsize{11}{1}\usefont{T1}{cmr}{m}{n}\selectfont\color{color_29791} }
\put(41.7,-532.2313){\fontsize{11}{1}\usefont{T1}{cmr}{m}{n}\selectfont\color{color_29791} }
\put(41.7,-546.8801){\fontsize{11}{1}\usefont{T1}{cmr}{m}{n}\selectfont\color{color_29791} }
\put(41.7,-561.5291){\fontsize{11}{1}\usefont{T1}{cmr}{m}{n}\selectfont\color{color_29791} }
\put(41.7,-576.1781){\fontsize{11}{1}\usefont{T1}{cmr}{m}{n}\selectfont\color{color_29791} }
\put(41.7,-590.8269){\fontsize{11}{1}\usefont{T1}{cmr}{m}{n}\selectfont\color{color_29791} }
\put(41.7,-605.4758){\fontsize{11}{1}\usefont{T1}{cmr}{m}{n}\selectfont\color{color_29791} }
\put(41.7,-620.1249){\fontsize{11}{1}\usefont{T1}{cmr}{m}{n}\selectfont\color{color_29791} }
\put(41.7,-634.7737){\fontsize{11}{1}\usefont{T1}{cmr}{m}{n}\selectfont\color{color_29791} }
\put(41.7,-649.4226){\fontsize{11}{1}\usefont{T1}{cmr}{m}{n}\selectfont\color{color_29791} }
\put(41.7,-664.0717){\fontsize{11}{1}\usefont{T1}{cmr}{m}{n}\selectfont\color{color_29791} }
\put(41.7,-678.7205){\fontsize{11}{1}\usefont{T1}{cmr}{m}{n}\selectfont\color{color_29791} }
\put(41.7,-693.3695){\fontsize{11}{1}\usefont{T1}{cmr}{m}{n}\selectfont\color{color_29791} }
\put(41.7,-708.0183){\fontsize{11}{1}\usefont{T1}{cmr}{m}{n}\selectfont\color{color_29791} }
\put(41.7,-722.6672){\fontsize{11}{1}\usefont{T1}{cmr}{m}{n}\selectfont\color{color_29791} }
\put(41.7,-737.3163){\fontsize{11}{1}\usefont{T1}{cmr}{m}{n}\selectfont\color{color_29791} }
\put(41.7,-751.9651){\fontsize{11}{1}\usefont{T1}{cmr}{m}{n}\selectfont\color{color_29791}       Cuiabá, MT }
\put(244.8522,-765.676){\fontsize{10}{1}\usefont{T1}{cmr}{m}{n}\selectfont\color{color_29791}Fevereiro – 2025 }
\end{picture}
\newpage
\begin{tikzpicture}[overlay]
\path(0pt,0pt);
\filldraw[color_283006][nonzero rule]
(-15pt, 10pt) -- (580.5pt, 10pt)
 -- (580.5pt, 10pt)
 -- (580.5pt, -832.25pt)
 -- (580.5pt, -832.25pt)
 -- (-15pt, -832.25pt) -- cycle
;
\end{tikzpicture}
\begin{picture}(-5,0)(2.5,0)
\put(254.3048,-57.95593){\fontsize{12}{1}\usefont{T1}{cmr}{b}{n}\selectfont\color{color_29791}SUMÁRIO }
\put(41.7,-73.17651){\fontsize{11}{1}\usefont{T1}{cmr}{b}{n}\selectfont\color{color_29791}1 INTRODUÇÃO 4 }
\put(41.7,-91.60425){\fontsize{11}{1}\usefont{T1}{cmr}{b}{n}\selectfont\color{color_29791}2 FERRAMENTAS E METODOLOGIA 4 }
\put(41.7,-110.032){\fontsize{11}{1}\usefont{T1}{cmr}{b}{n}\selectfont\color{color_29791}3 FUNDAMENTAÇÃO TEÓRICA 4 }
\put(52.95,-128.4597){\fontsize{11}{1}\usefont{T1}{cmr}{b}{n}\selectfont\color{color_29791}3.1 MIPS 4 }
\put(52.95,-153.6013){\fontsize{11}{1}\usefont{T1}{cmr}{b}{n}\selectfont\color{color_29791}3.2 Somador de 6 Bits 5 }
\put(52.95,-178.743){\fontsize{11}{1}\usefont{T1}{cmr}{b}{n}\selectfont\color{color_29791}3.3 Contador de Programa 5 }
\put(52.95,-203.8845){\fontsize{11}{1}\usefont{T1}{cmr}{b}{n}\selectfont\color{color_29791}3.4 Memória de Instruções (ROM) 5 }
\put(52.95,-229.0261){\fontsize{11}{1}\usefont{T1}{cmr}{b}{n}\selectfont\color{color_29791}3.5 Banco de Registradores (8x16) 5 }
\put(52.95,-254.1678){\fontsize{11}{1}\usefont{T1}{cmr}{b}{n}\selectfont\color{color_29791}3.6 Unidade Lógica e Aritmética (ULA) 5 }
\put(52.95,-278.9495){\fontsize{11}{1}\usefont{T1}{cmr}{b}{n}\selectfont\color{color_29791}3.7 Controlador da ULA }
\put(52.95,-303.2827){\fontsize{11}{1}\usefont{T1}{cmr}{b}{n}\selectfont\color{color_29791}3.8 Memória de Dados (RAM) 5 }
\put(52.95,-328.4244){\fontsize{11}{1}\usefont{T1}{cmr}{b}{n}\selectfont\color{color_29791}3.9 Extensor de Bits 5 }
\put(52.95,-353.5659){\fontsize{11}{1}\usefont{T1}{cmr}{b}{n}\selectfont\color{color_29791}3.10 Controlador/Sinais de Controle 6 }
\put(52.95,-378.7075){\fontsize{11}{1}\usefont{T1}{cmr}{b}{n}\selectfont\color{color_29791}3.11 Conjunto de Instruções 6 }
\put(41.7,-403.8491){\fontsize{11}{1}\usefont{T1}{cmr}{b}{n}\selectfont\color{color_29791}4 CIRCUITOS DESENVOLVIDOS 7 }
\put(41.7,-422.2769){\fontsize{11}{1}\usefont{T1}{cmr}{m}{n}\selectfont\color{color_29791}    4.1 Somador de 6 Bits 8 }
\put(41.7,-440.7046){\fontsize{11}{1}\usefont{T1}{cmr}{m}{n}\selectfont\color{color_29791}    4.2 Contador de Programa 8 }
\put(41.7,-459.1323){\fontsize{11}{1}\usefont{T1}{cmr}{m}{n}\selectfont\color{color_29791}    4.3 Memória de Instruções (ROM) 9  }
\put(41.7,-477.5601){\fontsize{11}{1}\usefont{T1}{cmr}{m}{n}\selectfont\color{color_29791}    4.4 Banco de Registradores (8x16) 10 }
\put(41.7,-495.9878){\fontsize{11}{1}\usefont{T1}{cmr}{m}{n}\selectfont\color{color_29791}    4.5 Unidade Lógica e Aritmética (ULA) 10 }
\put(41.7,-514.4155){\fontsize{11}{1}\usefont{T1}{cmr}{m}{n}\selectfont\color{color_29791}    4.6 Memória de Dados (RAM) 15 }
\put(41.7,-532.8433){\fontsize{11}{1}\usefont{T1}{cmr}{m}{n}\selectfont\color{color_29791}     4.7 Controlador da ULA }
\put(41.7,-551.271){\fontsize{11}{1}\usefont{T1}{cmr}{m}{n}\selectfont\color{color_29791}     4.8 Controlador de Sinais }
\put(41.7,-570.6694){\fontsize{11}{1}\usefont{T1}{cmr}{b}{n}\selectfont\color{color_29791}5 PROGRAMAS EM ASSEMBLY 10 }
\put(41.7,-589.3472){\fontsize{11}{1}\usefont{T1}{cmr}{b}{n}\selectfont\color{color_29791}6 CONCLUSÃO 15 }
\put(41.7,-607.7749){\fontsize{11}{1}\usefont{T1}{cmr}{b}{n}\selectfont\color{color_29791}7 REFERÊNCIAS 16 }
\end{picture}
\newpage
\begin{tikzpicture}[overlay]
\path(0pt,0pt);
\filldraw[color_283006][nonzero rule]
(-15pt, 10pt) -- (580.5pt, 10pt)
 -- (580.5pt, 10pt)
 -- (580.5pt, -832.25pt)
 -- (580.5pt, -832.25pt)
 -- (-15pt, -832.25pt) -- cycle
;
\end{tikzpicture}
\begin{picture}(-5,0)(2.5,0)
\put(41.7,-59.83203){\fontsize{14}{1}\usefont{T1}{cmr}{b}{n}\selectfont\color{color_29791} }
\put(41.7,-78.93066){\fontsize{14}{1}\usefont{T1}{cmr}{b}{n}\selectfont\color{color_29791} }
\put(282.6378,-98.0293){\fontsize{14}{1}\usefont{T1}{cmr}{b}{n}\selectfont\color{color_29791} }
\put(41.7,-117.1279){\fontsize{14}{1}\usefont{T1}{cmr}{b}{n}\selectfont\color{color_29791} }
\put(230.9425,-136.2266){\fontsize{14}{1}\usefont{T1}{cmr}{b}{n}\selectfont\color{color_29791}Criando o MIPS }
\put(41.7,-155.3252){\fontsize{14}{1}\usefont{T1}{cmr}{b}{n}\selectfont\color{color_29791} }
\put(282.6378,-174.4238){\fontsize{14}{1}\usefont{T1}{cmr}{b}{n}\selectfont\color{color_29791} }
\put(282.6378,-193.5225){\fontsize{14}{1}\usefont{T1}{cmr}{b}{n}\selectfont\color{color_29791} }
\put(41.7,-212.6211){\fontsize{14}{1}\usefont{T1}{cmr}{b}{n}\selectfont\color{color_29791} }
\put(282.6378,-228.9058){\fontsize{11}{1}\usefont{T1}{cmr}{m}{n}\selectfont\color{color_29791} }
\put(41.7,-245.4927){\fontsize{12}{1}\usefont{T1}{cmr}{b}{n}\selectfont\color{color_29791}1. INTRODUÇÃO }
\put(59.7,-262.2915){\fontsize{12}{1}\usefont{T1}{cmr}{b}{n}\selectfont\color{color_29791} }
\put(41.7,-279.0903){\fontsize{12}{1}\usefont{T1}{cmr}{m}{n}\selectfont\color{color_29791} A arquitetura MIPS (Microprocessor without Interlocked Pipeline Stages) }
\put(41.7,-292.8892){\fontsize{12}{1}\usefont{T1}{cmr}{m}{n}\selectfont\color{color_29791}destaca-se como uma das bases mais influentes no estudo e desenvolvimento de }
\put(41.7,-306.688){\fontsize{12}{1}\usefont{T1}{cmr}{m}{n}\selectfont\color{color_29791}processadores modernos. Originada na década de 1980, como parte do movimento RISC }
\put(41.7,-320.4868){\fontsize{12}{1}\usefont{T1}{cmr}{m}{n}\selectfont\color{color_29791}(Reduced Instruction Set Computer), a MIPS foi projetada para priorizar simplicidade e }
\put(41.7,-334.2856){\fontsize{12}{1}\usefont{T1}{cmr}{m}{n}\selectfont\color{color_29791}eficiência, reduzindo a complexidade do hardware e otimizando o desempenho por meio }
\put(41.7,-348.0845){\fontsize{12}{1}\usefont{T1}{cmr}{m}{n}\selectfont\color{color_29791}de um conjunto enxuto de instruções. Apesar das inovações que adicionaram recursos }
\put(41.7,-361.8833){\fontsize{12}{1}\usefont{T1}{cmr}{m}{n}\selectfont\color{color_29791}como pipeline avançado e execução fora de ordem, a essência minimalista do MIPS }
\put(41.7,-375.6821){\fontsize{12}{1}\usefont{T1}{cmr}{m}{n}\selectfont\color{color_29791}permanece amplamente utilizada como referência educacional. Sua clareza de design }
\put(41.7,-389.481){\fontsize{12}{1}\usefont{T1}{cmr}{m}{n}\selectfont\color{color_29791}torna-se especialmente útil para entender os fundamentos da arquitetura de }
\put(41.7,-403.2798){\fontsize{12}{1}\usefont{T1}{cmr}{m}{n}\selectfont\color{color_29791}computadores, incluindo a interação entre registradores, memória e unidades de controle. }
\put(41.7,-417.0786){\fontsize{12}{1}\usefont{T1}{cmr}{m}{n}\selectfont\color{color_29791}Este relatório concentra-se na implementação completa da arquitetura MIPS de 16 bits, }
\put(41.7,-430.8774){\fontsize{12}{1}\usefont{T1}{cmr}{m}{n}\selectfont\color{color_29791}com ênfase na integração do controlador principal, do ALUController e dos demais }
\put(41.7,-444.6763){\fontsize{12}{1}\usefont{T1}{cmr}{m}{n}\selectfont\color{color_29791}módulos que compõem o microprocessador. A análise detalha o funcionamento desses }
\put(41.7,-458.4751){\fontsize{12}{1}\usefont{T1}{cmr}{m}{n}\selectfont\color{color_29791}componentes, desde a decodificação de instruções até a execução de operações }
\put(41.7,-472.2739){\fontsize{12}{1}\usefont{T1}{cmr}{m}{n}\selectfont\color{color_29791}aritméticas e lógicas, passando pelo controle de barramentos e pela comunicação }
\put(41.7,-486.0728){\fontsize{12}{1}\usefont{T1}{cmr}{m}{n}\selectfont\color{color_29791}eficiente entre o banco de registradores e a memória RAM. O trabalho demonstra como a }
\put(41.7,-499.8716){\fontsize{12}{1}\usefont{T1}{cmr}{m}{n}\selectfont\color{color_29791}modularidade e a simplicidade do design MIPS permitem a execução precisa e eficiente }
\put(41.7,-513.6704){\fontsize{12}{1}\usefont{T1}{cmr}{m}{n}\selectfont\color{color_29791}de instruções, reforçando os princípios que sustentam sua relevância no ensino e no }
\put(41.7,-527.4692){\fontsize{12}{1}\usefont{T1}{cmr}{m}{n}\selectfont\color{color_29791}desenvolvimento de sistemas computacionais. }
\put(41.7,-543.2678){\fontsize{12}{1}\usefont{T1}{cmr}{m}{n}\selectfont\color{color_29791} }
\put(41.7,-559.0664){\fontsize{12}{1}\usefont{T1}{cmr}{b}{n}\selectfont\color{color_29791}2. FERRAMENTAS E METODOLOGIA }
\put(59.7,-574.8657){\fontsize{12}{1}\usefont{T1}{cmr}{b}{n}\selectfont\color{color_29791} }
\put(41.7,-590.6643){\fontsize{12}{1}\usefont{T1}{cmr}{m}{n}\selectfont\color{color_29791} Para a implementação parcial do MIPS, foi utilizado o Logisim, um software de }
\put(41.7,-604.4631){\fontsize{12}{1}\usefont{T1}{cmr}{m}{n}\selectfont\color{color_29791}código aberto amplamente reconhecido por sua eficácia na simulação de circuitos digitais. }
\put(41.7,-618.262){\fontsize{12}{1}\usefont{T1}{cmr}{m}{n}\selectfont\color{color_29791}A escolha pelo Logisim baseou-se em sua interface gráfica intuitiva, que facilita a criação }
\put(41.7,-632.0608){\fontsize{12}{1}\usefont{T1}{cmr}{m}{n}\selectfont\color{color_29791}e a visualização de sistemas digitais, tornando o processo acessível e didático. }
\put(41.7,-647.8594){\fontsize{12}{1}\usefont{T1}{cmr}{m}{n}\selectfont\color{color_29791} }
\put(41.7,-663.6587){\fontsize{12}{1}\usefont{T1}{cmr}{m}{n}\selectfont\color{color_29791} }
\put(41.7,-679.4573){\fontsize{12}{1}\usefont{T1}{cmr}{m}{n}\selectfont\color{color_29791} }
\put(41.7,-696.2559){\fontsize{12}{1}\usefont{T1}{cmr}{b}{n}\selectfont\color{color_29791}3. FUNDAMENTAÇÃO TEÓRICA }
\put(59.7,-713.0549){\fontsize{12}{1}\usefont{T1}{cmr}{b}{n}\selectfont\color{color_29791} }
\put(41.7,-729.8538){\fontsize{12}{1}\usefont{T1}{cmr}{b}{n}\selectfont\color{color_29791} O desenvolvimento do microprocessador MIPS exige a integração de diversos }
\put(41.7,-743.6526){\fontsize{12}{1}\usefont{T1}{cmr}{m}{n}\selectfont\color{color_29791}componentes lógicos e eletrônicos, que operam de forma sincronizada para executar }
\put(41.7,-757.4514){\fontsize{12}{1}\usefont{T1}{cmr}{m}{n}\selectfont\color{color_29791}instruções, manipular dados e gerenciar o fluxo de informações no sistema. A seguir, são }
\end{picture}
\newpage
\begin{tikzpicture}[overlay]
\path(0pt,0pt);
\filldraw[color_283006][nonzero rule]
(-15pt, 10pt) -- (580.5pt, 10pt)
 -- (580.5pt, 10pt)
 -- (580.5pt, -832.25pt)
 -- (580.5pt, -832.25pt)
 -- (-15pt, -832.25pt) -- cycle
;
\end{tikzpicture}
\begin{picture}(-5,0)(2.5,0)
\put(41.7,-57.95605){\fontsize{12}{1}\usefont{T1}{cmr}{m}{n}\selectfont\color{color_29791}apresentados os conceitos principais e os componentes essenciais que foram aplicados }
\put(41.7,-71.75488){\fontsize{12}{1}\usefont{T1}{cmr}{m}{n}\selectfont\color{color_29791}na construção dos circuitos que conectam o banco de registradores à memória RAM. }
\put(59.7,-88.55371){\fontsize{12}{1}\usefont{T1}{cmr}{m}{n}\selectfont\color{color_29791} }
\put(41.7,-105.3525){\fontsize{12}{1}\usefont{T1}{cmr}{b}{n}\selectfont\color{color_29791}3.1 MIPS }
\put(41.7,-122.1514){\fontsize{12}{1}\usefont{T1}{cmr}{b}{n}\selectfont\color{color_29791} }
\put(41.7,-138.9502){\fontsize{12}{1}\usefont{T1}{cmr}{b}{n}\selectfont\color{color_29791} O MIPS é uma arquitetura de processador amplamente utilizada para introduzir os }
\put(41.7,-152.749){\fontsize{12}{1}\usefont{T1}{cmr}{m}{n}\selectfont\color{color_29791}conceitos fundamentais da arquitetura de computadores. Ele foi projetado para executar }
\put(41.7,-166.5479){\fontsize{12}{1}\usefont{T1}{cmr}{m}{n}\selectfont\color{color_29791}instruções básicas, como leitura e escrita de dados, além de operações aritméticas e }
\put(41.7,-180.3467){\fontsize{12}{1}\usefont{T1}{cmr}{m}{n}\selectfont\color{color_29791}lógicas simples. Apesar de sua simplicidade em configurações educacionais, o MIPS }
\put(41.7,-194.1455){\fontsize{12}{1}\usefont{T1}{cmr}{m}{n}\selectfont\color{color_29791}incorpora elementos essenciais de um processador, incluindo banco de registradores, }
\put(41.7,-207.9443){\fontsize{12}{1}\usefont{T1}{cmr}{m}{n}\selectfont\color{color_29791}memória, barramentos e unidade de controle, oferecendo uma visão prática e clara sobre }
\put(41.7,-221.7432){\fontsize{12}{1}\usefont{T1}{cmr}{m}{n}\selectfont\color{color_29791}o funcionamento interno de sistemas digitais. }
\put(41.7,-238.542){\fontsize{12}{1}\usefont{T1}{cmr}{m}{n}\selectfont\color{color_29791} }
\put(41.7,-255.3408){\fontsize{12}{1}\usefont{T1}{cmr}{b}{n}\selectfont\color{color_29791} }
\put(41.7,-271.1394){\fontsize{12}{1}\usefont{T1}{cmr}{b}{n}\selectfont\color{color_29791}3.2 Somador de 6 Bits }
\put(41.7,-286.938){\fontsize{12}{1}\usefont{T1}{cmr}{b}{n}\selectfont\color{color_29791} }
\put(41.7,-302.7373){\fontsize{12}{1}\usefont{T1}{cmr}{b}{n}\selectfont\color{color_29791} O somador de 6 bits é um circuito usado para realizar a adição de dois números }
\put(41.7,-316.5361){\fontsize{12}{1}\usefont{T1}{cmr}{m}{n}\selectfont\color{color_29791}binários de até 6 bits. Ele trabalha propagando os valores de "carregamento" (carry) entre }
\put(41.7,-330.335){\fontsize{12}{1}\usefont{T1}{cmr}{m}{n}\selectfont\color{color_29791}os bits, garantindo que o resultado da soma esteja correto. Quando conectado ao }
\put(41.7,-344.1338){\fontsize{12}{1}\usefont{T1}{cmr}{m}{n}\selectfont\color{color_29791}contador de programa (PC) em sistemas baseados na arquitetura MIPS, o somador é }
\put(41.7,-357.9326){\fontsize{12}{1}\usefont{T1}{cmr}{m}{n}\selectfont\color{color_29791}responsável por incrementar o valor do PC, geralmente adicionando 1 para avançar para }
\put(41.7,-371.7314){\fontsize{12}{1}\usefont{T1}{cmr}{m}{n}\selectfont\color{color_29791}a próxima instrução na memória. Em instruções de desvio, o somador pode ser usado }
\put(41.7,-385.5303){\fontsize{12}{1}\usefont{T1}{cmr}{m}{n}\selectfont\color{color_29791}para calcular o novo endereço baseado em deslocamentos. Esse componente é essencial }
\put(41.7,-399.3291){\fontsize{12}{1}\usefont{T1}{cmr}{m}{n}\selectfont\color{color_29791}para o funcionamento do processador, permitindo que o fluxo de instruções seja }
\put(41.7,-413.1279){\fontsize{12}{1}\usefont{T1}{cmr}{m}{n}\selectfont\color{color_29791}controlado de forma eficiente e precisa. }
\put(41.7,-428.9265){\fontsize{12}{1}\usefont{T1}{cmr}{m}{n}\selectfont\color{color_29791} }
\put(41.7,-444.7251){\fontsize{12}{1}\usefont{T1}{cmr}{b}{n}\selectfont\color{color_29791}3.3 Contador de Programa }
\put(41.7,-460.5244){\fontsize{12}{1}\usefont{T1}{cmr}{b}{n}\selectfont\color{color_29791} }
\put(41.7,-476.323){\fontsize{12}{1}\usefont{T1}{cmr}{m}{n}\selectfont\color{color_29791} O contador de programa (PC - Program Counter) é um registrador essencial na }
\put(41.7,-490.1218){\fontsize{12}{1}\usefont{T1}{cmr}{m}{n}\selectfont\color{color_29791}arquitetura MIPS, responsável por armazenar o endereço da próxima instrução a ser }
\put(41.7,-503.9207){\fontsize{12}{1}\usefont{T1}{cmr}{m}{n}\selectfont\color{color_29791}executada. Ele atua como um controlador do fluxo de execução, garantindo que as }
\put(41.7,-517.7195){\fontsize{12}{1}\usefont{T1}{cmr}{m}{n}\selectfont\color{color_29791}instruções sejam lidas sequencialmente na maioria dos casos, ou desviadas para um }
\put(41.7,-531.5183){\fontsize{12}{1}\usefont{T1}{cmr}{m}{n}\selectfont\color{color_29791}endereço específico em situações de saltos ou chamadas de sub-rotinas. Em cada ciclo }
\put(41.7,-545.3171){\fontsize{12}{1}\usefont{T1}{cmr}{m}{n}\selectfont\color{color_29791}de clock, o PC é atualizado para apontar para a próxima instrução. Isso pode ser feito }
\put(41.7,-559.116){\fontsize{12}{1}\usefont{T1}{cmr}{m}{n}\selectfont\color{color_29791}incrementando seu valor por uma constante (geralmente 4, no caso de endereçamento }
\put(41.7,-572.9148){\fontsize{12}{1}\usefont{T1}{cmr}{m}{n}\selectfont\color{color_29791}por bytes, ou 1, no caso de endereçamento por palavras) ou, em instruções de desvio e }
\put(41.7,-586.7136){\fontsize{12}{1}\usefont{T1}{cmr}{m}{n}\selectfont\color{color_29791}salto, carregando um novo valor calculado com base em deslocamentos ou endereços }
\put(41.7,-600.5125){\fontsize{12}{1}\usefont{T1}{cmr}{m}{n}\selectfont\color{color_29791}imediatos. No MIPS, o PC interage diretamente com outros componentes, como a }
\put(41.7,-614.3113){\fontsize{12}{1}\usefont{T1}{cmr}{m}{n}\selectfont\color{color_29791}memória de instruções, que fornece a instrução localizada no endereço indicado pelo PC, }
\put(41.7,-628.1101){\fontsize{12}{1}\usefont{T1}{cmr}{m}{n}\selectfont\color{color_29791}e o somador, que calcula o próximo valor do contador. Essa interação é crucial para o }
\put(41.7,-641.9089){\fontsize{12}{1}\usefont{T1}{cmr}{m}{n}\selectfont\color{color_29791}funcionamento de instruções como branch (desvio condicional) e jump (salto }
\put(41.7,-655.7078){\fontsize{12}{1}\usefont{T1}{cmr}{m}{n}\selectfont\color{color_29791}incondicional), que alteram o fluxo de execução programado. }
\put(41.7,-681.5063){\fontsize{12}{1}\usefont{T1}{cmr}{b}{n}\selectfont\color{color_29791}3.4 Memória de Instruções (ROM) }
\put(41.7,-707.3057){\fontsize{12}{1}\usefont{T1}{cmr}{m}{n}\selectfont\color{color_29791} A memória de instruções no MIPS de 16 bits é responsável por armazenar o }
\put(41.7,-721.1045){\fontsize{12}{1}\usefont{T1}{cmr}{m}{n}\selectfont\color{color_29791}conjunto de instruções que o processador deve executar. Cada instrução é codificada em }
\put(41.7,-734.9033){\fontsize{12}{1}\usefont{T1}{cmr}{m}{n}\selectfont\color{color_29791}formato binário e ocupam 16 bits (2 bytes), alinhados na memória. Essa organização }
\put(41.7,-748.7021){\fontsize{12}{1}\usefont{T1}{cmr}{m}{n}\selectfont\color{color_29791}simplifica o acesso e a leitura das instruções pelo processador. Durante a execução, o }
\put(41.7,-762.501){\fontsize{12}{1}\usefont{T1}{cmr}{m}{n}\selectfont\color{color_29791}contador de programa (PC) fornece o endereço da próxima instrução à memória de }
\end{picture}
\newpage
\begin{tikzpicture}[overlay]
\path(0pt,0pt);
\filldraw[color_283006][nonzero rule]
(-15pt, 10pt) -- (580.5pt, 10pt)
 -- (580.5pt, 10pt)
 -- (580.5pt, -832.25pt)
 -- (580.5pt, -832.25pt)
 -- (-15pt, -832.25pt) -- cycle
;
\end{tikzpicture}
\begin{picture}(-5,0)(2.5,0)
\put(41.7,-57.95605){\fontsize{12}{1}\usefont{T1}{cmr}{m}{n}\selectfont\color{color_29791}instruções. A memória então retorna o código binário correspondente, que será }
\put(41.7,-71.75488){\fontsize{12}{1}\usefont{T1}{cmr}{m}{n}\selectfont\color{color_29791}decodificado e processado pela Unidade de Controle (UC). Esse ciclo, conhecido como }
\put(41.7,-85.55371){\fontsize{12}{1}\usefont{T1}{cmr}{m}{n}\selectfont\color{color_29791}busca de instrução, garante o funcionamento contínuo do sistema. No MIPS de 16 bits, a }
\put(41.7,-99.35254){\fontsize{12}{1}\usefont{T1}{cmr}{m}{n}\selectfont\color{color_29791}memória de instruções é tipicamente separada da memória de dados em arquiteturas }
\put(41.7,-113.1514){\fontsize{12}{1}\usefont{T1}{cmr}{m}{n}\selectfont\color{color_29791}Harvard, permitindo acessos paralelos, ou compartilhada em arquiteturas Von Neumann, }
\put(41.7,-126.9502){\fontsize{12}{1}\usefont{T1}{cmr}{m}{n}\selectfont\color{color_29791}dependendo do design adotado. A memória de instruções é um componente essencial }
\put(41.7,-140.749){\fontsize{12}{1}\usefont{T1}{cmr}{m}{n}\selectfont\color{color_29791}para o fluxo de execução, pois armazena e fornece as instruções necessárias de forma }
\put(41.7,-154.5479){\fontsize{12}{1}\usefont{T1}{cmr}{m}{n}\selectfont\color{color_29791}sequencial ou conforme indicado por desvios e saltos controlados pelo contador de }
\put(41.7,-168.3467){\fontsize{12}{1}\usefont{T1}{cmr}{m}{n}\selectfont\color{color_29791}programa. }
\put(41.7,-194.1455){\fontsize{12}{1}\usefont{T1}{cmr}{b}{n}\selectfont\color{color_29791}3.5 Banco de Registradores (8x16)  }
\put(41.7,-219.9443){\fontsize{12}{1}\usefont{T1}{cmr}{b}{n}\selectfont\color{color_29791} No processador MIPS, o Banco de Registradores é uma unidade fundamental que }
\put(41.7,-233.7432){\fontsize{12}{1}\usefont{T1}{cmr}{m}{n}\selectfont\color{color_29791}armazena dados temporários utilizados durante a execução de instruções. Ele consiste }
\put(41.7,-247.542){\fontsize{12}{1}\usefont{T1}{cmr}{m}{n}\selectfont\color{color_29791}em um conjunto de registradores de acesso rápido, diretamente integrados ao núcleo do }
\put(41.7,-261.3408){\fontsize{12}{1}\usefont{T1}{cmr}{m}{n}\selectfont\color{color_29791}processador, permitindo maior eficiência em comparação com a memória principal. }
\put(41.7,-275.1396){\fontsize{12}{1}\usefont{T1}{cmr}{m}{n}\selectfont\color{color_29791}Quando nos referimos a um "Banco de Registradores (8x16)", estamos tratando de um }
\put(41.7,-288.9385){\fontsize{12}{1}\usefont{T1}{cmr}{m}{n}\selectfont\color{color_29791}modelo composto por 8 registradores, cada um com capacidade para armazenar 16 bits }
\put(41.7,-302.7373){\fontsize{12}{1}\usefont{T1}{cmr}{m}{n}\selectfont\color{color_29791}de dados. Essa configuração é menor que a encontrada em implementações completas }
\put(41.7,-316.5361){\fontsize{12}{1}\usefont{T1}{cmr}{m}{n}\selectfont\color{color_29791}do MIPS, mas pode ser utilizada em exemplos simplificados ou modelos educacionais. Os }
\put(41.7,-330.335){\fontsize{12}{1}\usefont{T1}{cmr}{m}{n}\selectfont\color{color_29791}registradores desempenham funções essenciais, como armazenar operandos para }
\put(41.7,-344.1338){\fontsize{12}{1}\usefont{T1}{cmr}{m}{n}\selectfont\color{color_29791}operações aritméticas e lógicas, além de endereços ou resultados intermediários. O }
\put(41.7,-357.9326){\fontsize{12}{1}\usefont{T1}{cmr}{m}{n}\selectfont\color{color_29791}processador pode realizar operações simultâneas de leitura e escrita em dois }
\put(41.7,-371.7314){\fontsize{12}{1}\usefont{T1}{cmr}{m}{n}\selectfont\color{color_29791}registradores, dependendo da instrução. Por exemplo, em uma operação de soma, os }
\put(41.7,-385.5303){\fontsize{12}{1}\usefont{T1}{cmr}{m}{n}\selectfont\color{color_29791}valores podem ser carregados de dois registradores do banco, processados pela Unidade }
\put(41.7,-399.3291){\fontsize{12}{1}\usefont{T1}{cmr}{m}{n}\selectfont\color{color_29791}Lógica e Aritmética (ULA), e o resultado armazenado em outro registrador do mesmo }
\put(41.7,-413.1279){\fontsize{12}{1}\usefont{T1}{cmr}{m}{n}\selectfont\color{color_29791}banco. Essa estrutura é crucial para garantir o desempenho e a eficiência do processador }
\put(41.7,-426.9268){\fontsize{12}{1}\usefont{T1}{cmr}{m}{n}\selectfont\color{color_29791}durante a execução de programas. }
\put(41.7,-452.7256){\fontsize{12}{1}\usefont{T1}{cmr}{b}{n}\selectfont\color{color_29791}3.6 Unidade Lógica e Aritmética (ULA) }
\put(41.7,-478.5244){\fontsize{12}{1}\usefont{T1}{cmr}{b}{n}\selectfont\color{color_29791} A Unidade Lógica e Aritmética (ULA) é um componente essencial do processador }
\put(41.7,-492.3232){\fontsize{12}{1}\usefont{T1}{cmr}{m}{n}\selectfont\color{color_29791}MIPS, responsável por executar operações matemáticas e lógicas necessárias para o }
\put(41.7,-506.1221){\fontsize{12}{1}\usefont{T1}{cmr}{m}{n}\selectfont\color{color_29791}processamento de instruções. Ela interage diretamente com o Banco de Registradores e }
\put(41.7,-519.9209){\fontsize{12}{1}\usefont{T1}{cmr}{m}{n}\selectfont\color{color_29791}o caminho de dados, desempenhando um papel central na execução eficiente de }
\put(41.7,-533.7197){\fontsize{12}{1}\usefont{T1}{cmr}{m}{n}\selectfont\color{color_29791}programas. Entre suas principais funções estão as operações aritméticas, como soma, }
\put(41.7,-547.5186){\fontsize{12}{1}\usefont{T1}{cmr}{m}{n}\selectfont\color{color_29791}subtração, multiplicação e divisão, além das operações lógicas, como AND, OR, XOR e }
\put(41.7,-561.3174){\fontsize{12}{1}\usefont{T1}{cmr}{m}{n}\selectfont\color{color_29791}deslocamentos de bits (shift), essenciais para manipulação de dados binários. Além disso, }
\put(41.7,-575.1162){\fontsize{12}{1}\usefont{T1}{cmr}{m}{n}\selectfont\color{color_29791}a ULA é fundamental para a tomada de decisões, permitindo comparar valores e avaliar }
\put(41.7,-588.915){\fontsize{12}{1}\usefont{T1}{cmr}{m}{n}\selectfont\color{color_29791}condições, como igualdade ou desigualdade, o que é crucial para a execução de }
\put(41.7,-602.7139){\fontsize{12}{1}\usefont{T1}{cmr}{m}{n}\selectfont\color{color_29791}instruções condicionais, como saltos em programas. Seu funcionamento é controlado pelo }
\put(41.7,-616.5127){\fontsize{12}{1}\usefont{T1}{cmr}{m}{n}\selectfont\color{color_29791}controlador do processador, que determina qual operação a ULA deve realizar com base }
\put(41.7,-630.3115){\fontsize{12}{1}\usefont{T1}{cmr}{m}{n}\selectfont\color{color_29791}na instrução em execução. Por exemplo, em uma soma, os operandos são enviados pelo }
\put(41.7,-644.1104){\fontsize{12}{1}\usefont{T1}{cmr}{m}{n}\selectfont\color{color_29791}Banco de Registradores, a ULA realiza o cálculo, e o resultado é armazenado novamente }
\put(41.7,-657.9092){\fontsize{12}{1}\usefont{T1}{cmr}{m}{n}\selectfont\color{color_29791}no registrador ou usado em outra etapa. Projetada para oferecer alta velocidade e }
\put(41.7,-671.708){\fontsize{12}{1}\usefont{T1}{cmr}{m}{n}\selectfont\color{color_29791}precisão, a ULA é otimizada para garantir o desempenho eficiente do processador, sendo }
\put(41.7,-685.5068){\fontsize{12}{1}\usefont{T1}{cmr}{m}{n}\selectfont\color{color_29791}indispensável para a funcionalidade básica de qualquer arquitetura de CPU, incluindo o }
\put(41.7,-699.3057){\fontsize{12}{1}\usefont{T1}{cmr}{m}{n}\selectfont\color{color_29791}MIPS. }
\put(41.7,-725.1045){\fontsize{12}{1}\usefont{T1}{cmr}{b}{n}\selectfont\color{color_29791}3.7 Controlador da ULA }
\put(41.7,-750.9033){\fontsize{12}{1}\usefont{T1}{cmr}{m}{n}\selectfont\color{color_29791} O Controlador da ULA (ALUController) é um componente fundamental no }
\put(41.7,-764.7021){\fontsize{12}{1}\usefont{T1}{cmr}{m}{n}\selectfont\color{color_29791}microprocessador MIPS, responsável por gerenciar e coordenar as operações realizadas }
\end{picture}
\newpage
\begin{tikzpicture}[overlay]
\path(0pt,0pt);
\filldraw[color_283006][nonzero rule]
(-15pt, 10pt) -- (580.5pt, 10pt)
 -- (580.5pt, 10pt)
 -- (580.5pt, -832.25pt)
 -- (580.5pt, -832.25pt)
 -- (-15pt, -832.25pt) -- cycle
;
\end{tikzpicture}
\begin{picture}(-5,0)(2.5,0)
\put(41.7,-57.95605){\fontsize{12}{1}\usefont{T1}{cmr}{m}{n}\selectfont\color{color_29791}pela Unidade Lógica e Aritmética (ULA). Sua principal função é interpretar os sinais de }
\put(41.7,-71.75488){\fontsize{12}{1}\usefont{T1}{cmr}{m}{n}\selectfont\color{color_29791}controle enviados pela unidade de controle principal e convertê-los em comandos }
\put(41.7,-85.55371){\fontsize{12}{1}\usefont{T1}{cmr}{m}{n}\selectfont\color{color_29791}específicos para a ULA, assegurando a execução correta de operações aritméticas (como }
\put(41.7,-99.35254){\fontsize{12}{1}\usefont{T1}{cmr}{m}{n}\selectfont\color{color_29791}adição, subtração, multiplicação e divisão) e lógicas (como AND, OR, XOR e }
\put(41.7,-113.1514){\fontsize{12}{1}\usefont{T1}{cmr}{m}{n}\selectfont\color{color_29791}deslocamentos). O ALUController recebe como entrada os sinais de controle derivados do }
\put(41.7,-126.9502){\fontsize{12}{1}\usefont{T1}{cmr}{m}{n}\selectfont\color{color_29791}campo funct das instruções do tipo R (register) ou do opcode das instruções do tipo I }
\put(41.7,-140.749){\fontsize{12}{1}\usefont{T1}{cmr}{m}{n}\selectfont\color{color_29791}(immediate). Com base nesses sinais, ele gera comandos precisos, como o sinal ALUOp, }
\put(41.7,-154.5479){\fontsize{12}{1}\usefont{T1}{cmr}{m}{n}\selectfont\color{color_29791}que define a operação a ser realizada pela ULA. Além disso, o ALUController é }
\put(41.7,-168.3467){\fontsize{12}{1}\usefont{T1}{cmr}{m}{n}\selectfont\color{color_29791}responsável por mapear operações de alto nível (por exemplo, "realizar uma soma") em }
\put(41.7,-182.1455){\fontsize{12}{1}\usefont{T1}{cmr}{m}{n}\selectfont\color{color_29791}sinais de baixo nível que a ULA possa interpretar, utilizando lógica combinacional para }
\put(41.7,-195.9443){\fontsize{12}{1}\usefont{T1}{cmr}{m}{n}\selectfont\color{color_29791}traduzir os sinais de controle em ações específicas. Ele também suporta diferentes tipos }
\put(41.7,-209.7432){\fontsize{12}{1}\usefont{T1}{cmr}{m}{n}\selectfont\color{color_29791}de instruções, incluindo operações aritméticas e lógicas (tipo R), operações com valores }
\put(41.7,-223.542){\fontsize{12}{1}\usefont{T1}{cmr}{m}{n}\selectfont\color{color_29791}imediatos (tipo I) e instruções de deslocamento. Integrado à ULA, o ALUController garante }
\put(41.7,-237.3408){\fontsize{12}{1}\usefont{T1}{cmr}{m}{n}\selectfont\color{color_29791}que os operandos fornecidos pelo banco de registradores ou pela memória sejam }
\put(41.7,-251.1396){\fontsize{12}{1}\usefont{T1}{cmr}{m}{n}\selectfont\color{color_29791}processados de maneira eficiente e precisa, desempenhando um papel crucial na }
\put(41.7,-264.9385){\fontsize{12}{1}\usefont{T1}{cmr}{m}{n}\selectfont\color{color_29791}execução das instruções e no fluxo de dados do microprocessador }
\put(41.7,-290.7373){\fontsize{12}{1}\usefont{T1}{cmr}{b}{n}\selectfont\color{color_29791}3.8 Memória de Dados (RAM)          }
\put(41.7,-316.5361){\fontsize{12}{1}\usefont{T1}{cmr}{m}{n}\selectfont\color{color_29791} A Memória de Dados (RAM) no processador MIPS é uma unidade essencial para o }
\put(41.7,-330.335){\fontsize{12}{1}\usefont{T1}{cmr}{m}{n}\selectfont\color{color_29791}armazenamento temporário de informações necessárias durante a execução de }
\put(41.7,-344.1338){\fontsize{12}{1}\usefont{T1}{cmr}{m}{n}\selectfont\color{color_29791}programas. Essa memória é do tipo volátil, ou seja, seus dados são perdidos quando o }
\put(41.7,-357.9326){\fontsize{12}{1}\usefont{T1}{cmr}{m}{n}\selectfont\color{color_29791}sistema é desligado, e é usada para guardar variáveis, dados intermediários e resultados }
\put(41.7,-371.7314){\fontsize{12}{1}\usefont{T1}{cmr}{m}{n}\selectfont\color{color_29791}que precisam ser acessados rapidamente pelo processador. No contexto do MIPS, a RAM }
\put(41.7,-385.5303){\fontsize{12}{1}\usefont{T1}{cmr}{m}{n}\selectfont\color{color_29791}é organizada em endereços, permitindo que o processador leia ou escreva dados }
\put(41.7,-399.3291){\fontsize{12}{1}\usefont{T1}{cmr}{m}{n}\selectfont\color{color_29791}específicos conforme necessário. Por exemplo, instruções de carga (load) e }
\put(41.7,-413.1279){\fontsize{12}{1}\usefont{T1}{cmr}{m}{n}\selectfont\color{color_29791}armazenamento (store) utilizam a memória para transferir informações entre a RAM e o }
\put(41.7,-426.9268){\fontsize{12}{1}\usefont{T1}{cmr}{m}{n}\selectfont\color{color_29791}Banco de Registradores. A RAM desempenha um papel crucial na execução de }
\put(41.7,-440.7256){\fontsize{12}{1}\usefont{T1}{cmr}{m}{n}\selectfont\color{color_29791}programas, permitindo o armazenamento temporário de instruções ou dados que não }
\put(41.7,-454.5244){\fontsize{12}{1}\usefont{T1}{cmr}{m}{n}\selectfont\color{color_29791}cabem nos registradores internos. Sua capacidade e velocidade têm impacto direto no }
\put(41.7,-468.3232){\fontsize{12}{1}\usefont{T1}{cmr}{m}{n}\selectfont\color{color_29791}desempenho do sistema, pois uma memória rápida reduz os atrasos no acesso aos }
\put(41.7,-482.1221){\fontsize{12}{1}\usefont{T1}{cmr}{m}{n}\selectfont\color{color_29791}dados, enquanto uma maior capacidade suporta programas mais complexos e que }
\put(41.7,-495.9209){\fontsize{12}{1}\usefont{T1}{cmr}{m}{n}\selectfont\color{color_29791}requerem mais espaço de armazenamento temporário. Assim, a Memória de Dados é }
\put(41.7,-509.7197){\fontsize{12}{1}\usefont{T1}{cmr}{m}{n}\selectfont\color{color_29791}indispensável para garantir o funcionamento eficiente do processador MIPS e de sistemas }
\put(41.7,-523.5186){\fontsize{12}{1}\usefont{T1}{cmr}{m}{n}\selectfont\color{color_29791}computacionais em geral. }
\put(41.7,-549.3174){\fontsize{12}{1}\usefont{T1}{cmr}{b}{n}\selectfont\color{color_29791}3.9 Extensor de Bits }
\put(41.7,-575.1162){\fontsize{12}{1}\usefont{T1}{cmr}{b}{n}\selectfont\color{color_29791} O Extensor de Bits é um componente do processador MIPS que ajusta o tamanho }
\put(41.7,-588.915){\fontsize{12}{1}\usefont{T1}{cmr}{m}{n}\selectfont\color{color_29791}de dados binários para compatibilidade com o tamanho padrão dos registradores, }
\put(41.7,-602.7139){\fontsize{12}{1}\usefont{T1}{cmr}{m}{n}\selectfont\color{color_29791}garantindo a execução correta de operações. Ele é usado principalmente para expandir }
\put(41.7,-616.5127){\fontsize{12}{1}\usefont{T1}{cmr}{m}{n}\selectfont\color{color_29791}valores menores, como números de 6 bits, para 16 bits, quando são utilizados em }
\put(41.7,-630.3115){\fontsize{12}{1}\usefont{T1}{cmr}{m}{n}\selectfont\color{color_29791}instruções imediatas ou ao carregar dados da memória. O processo pode ser realizado de }
\put(41.7,-644.1104){\fontsize{12}{1}\usefont{T1}{cmr}{m}{n}\selectfont\color{color_29791}duas formas: na extensão com sinal (sign-extension), o bit mais significativo do número }
\put(41.7,-657.9092){\fontsize{12}{1}\usefont{T1}{cmr}{m}{n}\selectfont\color{color_29791}original é replicado nos bits adicionais para preservar o sinal (positivo ou negativo); já na }
\put(41.7,-671.708){\fontsize{12}{1}\usefont{T1}{cmr}{m}{n}\selectfont\color{color_29791}extensão sem sinal (zero-extension), os bits extras são preenchidos com zeros. Essa }
\put(41.7,-685.5068){\fontsize{12}{1}\usefont{T1}{cmr}{m}{n}\selectfont\color{color_29791}funcionalidade é essencial para evitar erros ao operar com diferentes tamanhos de dados }
\put(41.7,-699.3057){\fontsize{12}{1}\usefont{T1}{cmr}{m}{n}\selectfont\color{color_29791}e assegurar que cálculos e transferências sejam feitos corretamente dentro do }
\put(41.7,-713.1045){\fontsize{12}{1}\usefont{T1}{cmr}{m}{n}\selectfont\color{color_29791}processador. }
\put(41.7,-738.9033){\fontsize{12}{1}\usefont{T1}{cmr}{b}{n}\selectfont\color{color_29791}3.10 Controlador/Sinais de Controle }
\end{picture}
\newpage
\begin{tikzpicture}[overlay]
\path(0pt,0pt);
\filldraw[color_283006][nonzero rule]
(-15pt, 10pt) -- (580.5pt, 10pt)
 -- (580.5pt, 10pt)
 -- (580.5pt, -832.25pt)
 -- (580.5pt, -832.25pt)
 -- (-15pt, -832.25pt) -- cycle
;
\end{tikzpicture}
\begin{picture}(-5,0)(2.5,0)
\put(41.7,-57.95605){\fontsize{12}{1}\usefont{T1}{cmr}{m}{n}\selectfont\color{color_29791} Os Sinais de Controle são essenciais no processador MIPS, pois coordenam o }
\put(41.7,-71.75488){\fontsize{12}{1}\usefont{T1}{cmr}{m}{n}\selectfont\color{color_29791}funcionamento de seus diversos componentes, garantindo que as instruções sejam }
\put(41.7,-85.55371){\fontsize{12}{1}\usefont{T1}{cmr}{m}{n}\selectfont\color{color_29791}executadas corretamente. No projeto inicial, esses sinais eram gerados manualmente, }
\put(41.7,-99.35254){\fontsize{12}{1}\usefont{T1}{cmr}{m}{n}\selectfont\color{color_29791}mas, na implementação final, eles passaram a ser gerenciados pelo Controlador Geral, }
\put(41.7,-113.1514){\fontsize{12}{1}\usefont{T1}{cmr}{m}{n}\selectfont\color{color_29791}uma unidade centralizada que automatiza e sincroniza o fluxo de execução das }
\put(41.7,-126.9502){\fontsize{12}{1}\usefont{T1}{cmr}{m}{n}\selectfont\color{color_29791}instruções. O Controlador Geral é responsável por interpretar o opcode e os campos }
\put(41.7,-140.749){\fontsize{12}{1}\usefont{T1}{cmr}{m}{n}\selectfont\color{color_29791}adicionais da instrução, como o funct e os registradores envolvidos, e gerar os sinais de }
\put(41.7,-154.5479){\fontsize{12}{1}\usefont{T1}{cmr}{m}{n}\selectfont\color{color_29791}controle necessários para operar a ULA, o Banco de Registradores, a Memória, o }
\put(41.7,-168.3467){\fontsize{12}{1}\usefont{T1}{cmr}{m}{n}\selectfont\color{color_29791}Extensor de Bits e outros módulos do processador. Esses sinais determinam ações }
\put(41.7,-182.1455){\fontsize{12}{1}\usefont{T1}{cmr}{m}{n}\selectfont\color{color_29791}específicas, como a seleção de registradores para leitura ou escrita, o tipo de operação }
\put(41.7,-195.9438){\fontsize{12}{1}\usefont{T1}{cmr}{m}{n}\selectfont\color{color_29791}que a ULA deve executar, e se os dados devem ser carregados da memória ou }
\put(41.7,-209.7427){\fontsize{12}{1}\usefont{T1}{cmr}{m}{n}\selectfont\color{color_29791}armazenados nela. Por exemplo, durante uma instrução de soma, o Controlador configura }
\put(41.7,-223.5415){\fontsize{12}{1}\usefont{T1}{cmr}{m}{n}\selectfont\color{color_29791}a ULA para realizar a operação aritmética, define quais registradores fornecerão os }
\put(41.7,-237.3403){\fontsize{12}{1}\usefont{T1}{cmr}{m}{n}\selectfont\color{color_29791}operandos e onde o resultado será armazenado. Além disso, o Controlador Geral }
\put(41.7,-251.1392){\fontsize{12}{1}\usefont{T1}{cmr}{m}{n}\selectfont\color{color_29791}gerencia operações condicionais, como saltos (branch), e configura multiplexadores para }
\put(41.7,-264.938){\fontsize{12}{1}\usefont{T1}{cmr}{m}{n}\selectfont\color{color_29791}selecionar os caminhos de dados corretos. Assim, ele funciona como o "sistema nervoso" }
\put(41.7,-278.7368){\fontsize{12}{1}\usefont{T1}{cmr}{m}{n}\selectfont\color{color_29791}do processador, orquestrando todas as operações de maneira sincronizada e garantindo }
\put(41.7,-292.5356){\fontsize{12}{1}\usefont{T1}{cmr}{m}{n}\selectfont\color{color_29791}que as instruções sejam executadas com precisão. }
\put(59.7,-318.3345){\fontsize{12}{1}\usefont{T1}{cmr}{m}{n}\selectfont\color{color_29791}1. RegDst: Define qual registro destino será utilizado (por exemplo, o registrador rt ou }
\put(77.7,-332.1333){\fontsize{12}{1}\usefont{T1}{cmr}{m}{n}\selectfont\color{color_29791}o registrador rd). }
\put(59.7,-345.9321){\fontsize{12}{1}\usefont{T1}{cmr}{m}{n}\selectfont\color{color_29791}2. CLK: O sinal de relógio controla quando as operações ocorrem no processador, }
\put(77.7,-359.731){\fontsize{12}{1}\usefont{T1}{cmr}{m}{n}\selectfont\color{color_29791}sincronizando as ações. }
\put(59.7,-373.5298){\fontsize{12}{1}\usefont{T1}{cmr}{m}{n}\selectfont\color{color_29791}3. RegWrite: Permite ou impede a escrita de novos valores nos registradores. }
\put(59.7,-387.3286){\fontsize{12}{1}\usefont{T1}{cmr}{m}{n}\selectfont\color{color_29791}4. CLR: Serve para limpar determinados componentes do processador, como }
\put(77.7,-401.1274){\fontsize{12}{1}\usefont{T1}{cmr}{m}{n}\selectfont\color{color_29791}registradores e a memória. }
\put(59.7,-414.9263){\fontsize{12}{1}\usefont{T1}{cmr}{m}{n}\selectfont\color{color_29791}5. PC\_Enable: Habilita ou desabilita o carregamento do contador de programa, }
\put(77.7,-428.7251){\fontsize{12}{1}\usefont{T1}{cmr}{m}{n}\selectfont\color{color_29791}controlando a sequência de execução das instruções. }
\put(59.7,-442.5239){\fontsize{12}{1}\usefont{T1}{cmr}{m}{n}\selectfont\color{color_29791}6. ALUsrc: Define se a entrada da unidade aritmético-lógica (ALU) será um operando }
\put(77.7,-456.3228){\fontsize{12}{1}\usefont{T1}{cmr}{m}{n}\selectfont\color{color_29791}imediato ou um registrador. }
\put(59.7,-470.1216){\fontsize{12}{1}\usefont{T1}{cmr}{m}{n}\selectfont\color{color_29791}7. MemWrite: Controla se as operações de escrita na memória estão ativas. }
\put(59.7,-483.9204){\fontsize{12}{1}\usefont{T1}{cmr}{m}{n}\selectfont\color{color_29791}8. MemRead: Controla se as operações de leitura na memória estão ativas. }
\put(59.7,-497.7192){\fontsize{12}{1}\usefont{T1}{cmr}{m}{n}\selectfont\color{color_29791}9. MemToReg: Define de onde o resultado da memória deve ser encaminhado }
\put(77.7,-511.5181){\fontsize{12}{1}\usefont{T1}{cmr}{m}{n}\selectfont\color{color_29791}(memória ou registrador). }
\put(59.7,-525.3169){\fontsize{12}{1}\usefont{T1}{cmr}{m}{n}\selectfont\color{color_29791}10. Jump: Controla a execução de instruções de salto incondicional (jump), }
\put(77.7,-539.1157){\fontsize{12}{1}\usefont{T1}{cmr}{m}{n}\selectfont\color{color_29791}direcionando o contador de programa (PC) para um endereço específico. }
\put(59.7,-552.9146){\fontsize{12}{1}\usefont{T1}{cmr}{m}{n}\selectfont\color{color_29791}11. Branch: Gerencia instruções de desvio condicional (branch), ativando o sinal de }
\put(77.7,-566.7134){\fontsize{12}{1}\usefont{T1}{cmr}{m}{n}\selectfont\color{color_29791}Branch quando uma condição é atendida (por exemplo, se dois registradores são }
\put(77.7,-580.5122){\fontsize{12}{1}\usefont{T1}{cmr}{m}{n}\selectfont\color{color_29791}iguais) e redirecionando o fluxo de execução. }
\put(59.7,-594.311){\fontsize{12}{1}\usefont{T1}{cmr}{m}{n}\selectfont\color{color_29791}12. Halt: Interrompe a execução do processador, finalizando o ciclo de instruções e }
\put(77.7,-608.1099){\fontsize{12}{1}\usefont{T1}{cmr}{m}{n}\selectfont\color{color_29791}parando o funcionamento do sistema. }
\put(41.7,-633.9087){\fontsize{12}{1}\usefont{T1}{cmr}{m}{n}\selectfont\color{color_29791}Com a implementação do Controlador Geral, o processador MIPS ganhou maior }
\put(41.7,-647.7075){\fontsize{12}{1}\usefont{T1}{cmr}{m}{n}\selectfont\color{color_29791}autonomia e eficiência, eliminando a necessidade de controle manual dos sinais e }
\put(41.7,-661.5063){\fontsize{12}{1}\usefont{T1}{cmr}{m}{n}\selectfont\color{color_29791}permitindo a execução automatizada e precisa das instruções. O Controlador Geral, em }
\put(41.7,-675.3052){\fontsize{12}{1}\usefont{T1}{cmr}{m}{n}\selectfont\color{color_29791}conjunto com o ALUController, forma o núcleo de gerenciamento do processador, }
\put(41.7,-689.104){\fontsize{12}{1}\usefont{T1}{cmr}{m}{n}\selectfont\color{color_29791}garantindo que todas as operações sejam realizadas de acordo com a arquitetura MIPS e }
\put(41.7,-702.9028){\fontsize{12}{1}\usefont{T1}{cmr}{m}{n}\selectfont\color{color_29791}seus princípios de simplicidade e modularidade. }
\put(41.7,-728.7017){\fontsize{12}{1}\usefont{T1}{cmr}{b}{n}\selectfont\color{color_29791}3.11 Conjunto de Instruções }
\put(41.7,-754.5005){\fontsize{12}{1}\usefont{T1}{cmr}{b}{n}\selectfont\color{color_29791} O conjunto de instruções do processador MIPS de 16 bits é baseado em uma }
\put(41.7,-768.2993){\fontsize{12}{1}\usefont{T1}{cmr}{m}{n}\selectfont\color{color_29791}arquitetura RISC (Reduced Instruction Set Computer), que prioriza simplicidade e }
\end{picture}
\newpage
\begin{tikzpicture}[overlay]
\path(0pt,0pt);
\filldraw[color_283006][nonzero rule]
(-15pt, 10pt) -- (580.5pt, 10pt)
 -- (580.5pt, 10pt)
 -- (580.5pt, -832.25pt)
 -- (580.5pt, -832.25pt)
 -- (-15pt, -832.25pt) -- cycle
;
\end{tikzpicture}
\begin{picture}(-5,0)(2.5,0)
\put(41.7,-57.95557){\fontsize{12}{1}\usefont{T1}{cmr}{m}{n}\selectfont\color{color_29791}eficiência. As instruções possuem tamanho fixo de 16 bits, o que facilita a decodificação e }
\put(41.7,-71.75439){\fontsize{12}{1}\usefont{T1}{cmr}{m}{n}\selectfont\color{color_29791}execução no hardware. Esse conjunto é composto por três formatos principais: R, I e J, }
\put(41.7,-85.55322){\fontsize{12}{1}\usefont{T1}{cmr}{m}{n}\selectfont\color{color_29791}cada um projetado para atender a diferentes tipos de operações. }
\put(59.7,-111.3521){\fontsize{12}{1}\usefont{T1}{cmr}{m}{n}\selectfont\color{color_29791}● Instruções do Tipo R: São usadas para operações aritméticas e lógicas entre }
\put(77.7,-125.1509){\fontsize{12}{1}\usefont{T1}{cmr}{m}{n}\selectfont\color{color_29791}registradores, como soma, subtração e operações lógicas (AND, OR, etc.). O }
\put(77.7,-138.9497){\fontsize{12}{1}\usefont{T1}{cmr}{m}{n}\selectfont\color{color_29791}formato inclui campos para especificar até dois registradores de entrada, um }
\put(77.7,-152.7485){\fontsize{12}{1}\usefont{T1}{cmr}{m}{n}\selectfont\color{color_29791}registrador de destino e um campo para o código da função (funct), além do }
\put(77.7,-166.5474){\fontsize{12}{1}\usefont{T1}{cmr}{m}{n}\selectfont\color{color_29791}opcode que identifica a instrução. Exemplos: add, sub, and, or. }
\put(59.7,-180.3462){\fontsize{12}{1}\usefont{T1}{cmr}{m}{n}\selectfont\color{color_29791}● Instruções do Tipo I: São destinadas a operações que envolvem um registrador e }
\put(77.7,-194.145){\fontsize{12}{1}\usefont{T1}{cmr}{m}{n}\selectfont\color{color_29791}um valor imediato, ou acesso à memória. O formato possui campos para dois }
\put(77.7,-207.9438){\fontsize{12}{1}\usefont{T1}{cmr}{m}{n}\selectfont\color{color_29791}registradores (um de base e um de destino), um valor imediato de 5 bits e o }
\put(77.7,-221.7427){\fontsize{12}{1}\usefont{T1}{cmr}{m}{n}\selectfont\color{color_29791}opcode. Exemplos incluem: lw (load word), sw (store word), addi (add immediate) e }
\put(77.7,-235.5415){\fontsize{12}{1}\usefont{T1}{cmr}{m}{n}\selectfont\color{color_29791}instruções de comparação. }
\put(59.7,-249.3403){\fontsize{12}{1}\usefont{T1}{cmr}{m}{n}\selectfont\color{color_29791}● Instruções do Tipo J: São utilizadas para saltos de longo alcance, permitindo }
\put(77.7,-263.1392){\fontsize{12}{1}\usefont{T1}{cmr}{m}{n}\selectfont\color{color_29791}alterar o fluxo do programa para uma posição específica na memória. O formato }
\put(77.7,-276.938){\fontsize{12}{1}\usefont{T1}{cmr}{m}{n}\selectfont\color{color_29791}contém um opcode e um endereço de 11 bits, possibilitando saltos dentro do }
\put(77.7,-290.7368){\fontsize{12}{1}\usefont{T1}{cmr}{m}{n}\selectfont\color{color_29791}espaço de endereçamento de 16 bits. Exemplos: j (jump) e jal (jump and link). }
\put(41.7,-316.5356){\fontsize{12}{1}\usefont{T1}{cmr}{m}{n}\selectfont\color{color_29791}No MIPS de 16 bits, cada instrução é representada em linguagem assembly, que utiliza }
\put(41.7,-330.3345){\fontsize{12}{1}\usefont{T1}{cmr}{m}{n}\selectfont\color{color_29791}uma notação específica para definir as operações e seus operandos. Essa linguagem se }
\put(41.7,-344.1333){\fontsize{12}{1}\usefont{T1}{cmr}{m}{n}\selectfont\color{color_29791}baseia em mnemônicos (como add, sub, lw, sw) seguidos dos registradores ou valores }
\put(41.7,-357.9321){\fontsize{12}{1}\usefont{T1}{cmr}{m}{n}\selectfont\color{color_29791}constantes utilizados. Na arquitetura MIPS de 16 bits, a notação para registradores utiliza }
\put(41.7,-371.731){\fontsize{12}{1}\usefont{T1}{cmr}{m}{n}\selectfont\color{color_29791}o símbolo \$ e um número que identifica o registrador. Neste caso, há 8 registradores }
\put(41.7,-385.5298){\fontsize{12}{1}\usefont{T1}{cmr}{m}{n}\selectfont\color{color_29791}disponíveis, que vão de \$0 a \$7. Os registradores são usados para armazenar }
\put(41.7,-399.3286){\fontsize{12}{1}\usefont{T1}{cmr}{m}{n}\selectfont\color{color_29791}temporariamente dados e endereços durante a execução das instruções. Quando se trata }
\put(41.7,-413.1274){\fontsize{12}{1}\usefont{T1}{cmr}{m}{n}\selectfont\color{color_29791}de valores imediatos ou constantes, estes são escritos diretamente, sem nenhum símbolo }
\put(41.7,-426.9263){\fontsize{12}{1}\usefont{T1}{cmr}{m}{n}\selectfont\color{color_29791}adicional antes ou depois. Já para instruções que acessam a memória, como lw (load }
\put(41.7,-440.7251){\fontsize{12}{1}\usefont{T1}{cmr}{m}{n}\selectfont\color{color_29791}word) e sw (store word), a sintaxe é diferente. Nessas instruções, o endereço de memória }
\put(41.7,-454.5239){\fontsize{12}{1}\usefont{T1}{cmr}{m}{n}\selectfont\color{color_29791}a ser acessado é especificado como a soma de uma constante e o valor armazenado em }
\put(41.7,-468.3228){\fontsize{12}{1}\usefont{T1}{cmr}{m}{n}\selectfont\color{color_29791}um registrador, utilizando parênteses para delimitar o registrador. Por exemplo, na }
\put(41.7,-482.1216){\fontsize{12}{1}\usefont{T1}{cmr}{m}{n}\selectfont\color{color_29791}instrução add \$1, \$3, \$2, que é do tipo R, a operação realizada será \$2 = \$1 + \$3, ou seja, }
\put(41.7,-495.9204){\fontsize{12}{1}\usefont{T1}{cmr}{m}{n}\selectfont\color{color_29791}o registrador \$2 receberá a soma dos conteúdos de \$1 e \$3. Já na instrução lw \$7, 1(\$3), }
\put(41.7,-509.7192){\fontsize{12}{1}\usefont{T1}{cmr}{m}{n}\selectfont\color{color_29791}que é do tipo I, o registrador \$7 carregará o valor armazenado na posição de memória }
\put(41.7,-523.5181){\fontsize{12}{1}\usefont{T1}{cmr}{m}{n}\selectfont\color{color_29791}correspondente à soma de 1 com o conteúdo de \$3. Assim, se \$3 contiver o valor 7, \$7 }
\put(41.7,-537.3169){\fontsize{12}{1}\usefont{T1}{cmr}{m}{n}\selectfont\color{color_29791}receberá o dado da posição de memória 8 (7 + 1). }
\put(41.7,-563.1157){\fontsize{12}{1}\usefont{T1}{cmr}{m}{n}\selectfont\color{color_29791}Ademais, para que o computador possa executar as instruções escritas em assembly, }
\put(41.7,-576.9146){\fontsize{12}{1}\usefont{T1}{cmr}{m}{n}\selectfont\color{color_29791}elas precisam ser convertidas para o formato binário, conhecido como código de máquina. }
\put(41.7,-590.7134){\fontsize{12}{1}\usefont{T1}{cmr}{m}{n}\selectfont\color{color_29791}Esse processo traduz as instruções para um padrão que o hardware do processador }
\put(41.7,-604.5122){\fontsize{12}{1}\usefont{T1}{cmr}{m}{n}\selectfont\color{color_29791}MIPS pode entender e executar. No MIPS de 16 bits, cada instrução ocupa exatamente }
\put(41.7,-618.311){\fontsize{12}{1}\usefont{T1}{cmr}{m}{n}\selectfont\color{color_29791}16 bits e é organizada em palavras, cuja estrutura varia dependendo do formato da }
\put(41.7,-632.1099){\fontsize{12}{1}\usefont{T1}{cmr}{m}{n}\selectfont\color{color_29791}instrução: R, I ou J. No formato R, usado para operações aritméticas e lógicas entre }
\put(41.7,-645.9087){\fontsize{12}{1}\usefont{T1}{cmr}{m}{n}\selectfont\color{color_29791}registradores, a palavra de 16 bits é dividida em 4 bits para o opcode, que identifica a }
\put(41.7,-659.7075){\fontsize{12}{1}\usefont{T1}{cmr}{m}{n}\selectfont\color{color_29791}operação, 3 bits para o registrador fonte (rs), 3 bits para o registrador alvo (rt), 3 bits para }
\put(41.7,-673.5063){\fontsize{12}{1}\usefont{T1}{cmr}{m}{n}\selectfont\color{color_29791}o registrador de destino (rd) e 3 bits para o campo funct, que detalha a função específica }
\put(41.7,-687.3052){\fontsize{12}{1}\usefont{T1}{cmr}{m}{n}\selectfont\color{color_29791}a ser executada. Por exemplo, a instrução add \$0, \$1, \$2 seria codificada atribuindo os }
\put(41.7,-701.104){\fontsize{12}{1}\usefont{T1}{cmr}{m}{n}\selectfont\color{color_29791}valores correspondentes aos campos acima. No formato I, usado para operações com }
\put(41.7,-714.9028){\fontsize{12}{1}\usefont{T1}{cmr}{m}{n}\selectfont\color{color_29791}valores imediatos ou acesso à memória, a divisão da palavra é feita em 4 bits para o }
\put(41.7,-728.7017){\fontsize{12}{1}\usefont{T1}{cmr}{m}{n}\selectfont\color{color_29791}opcode, 3 bits para o registrador fonte (rs), 3 bits para o registrador destino (rt) e 6 bits }
\put(41.7,-742.5005){\fontsize{12}{1}\usefont{T1}{cmr}{m}{n}\selectfont\color{color_29791}para o valor imediato. Por exemplo, a instrução addi \$1, \$2, 15 será codificada com os }
\put(41.7,-756.2993){\fontsize{12}{1}\usefont{T1}{cmr}{m}{n}\selectfont\color{color_29791}valores do opcode, registradores e o imediato no formato correspondente. No formato J, }
\put(41.7,-770.0981){\fontsize{12}{1}\usefont{T1}{cmr}{m}{n}\selectfont\color{color_29791}usado para saltos no fluxo de execução, a organização é composta por 4 bits para o }
\end{picture}
\newpage
\begin{tikzpicture}[overlay]
\path(0pt,0pt);
\filldraw[color_283006][nonzero rule]
(-15pt, 10pt) -- (580.5pt, 10pt)
 -- (580.5pt, 10pt)
 -- (580.5pt, -832.25pt)
 -- (580.5pt, -832.25pt)
 -- (-15pt, -832.25pt) -- cycle
;
\end{tikzpicture}
\begin{picture}(-5,0)(2.5,0)
\put(41.7,-57.95557){\fontsize{12}{1}\usefont{T1}{cmr}{m}{n}\selectfont\color{color_29791}opcode e 12 bits para o endereço de destino. Por exemplo, a instrução j 50 será }
\put(41.7,-71.75439){\fontsize{12}{1}\usefont{T1}{cmr}{m}{n}\selectfont\color{color_29791}representada com o opcode e o endereço de destino apropriados. }
\put(41.7,-97.55322){\fontsize{12}{1}\usefont{T1}{cmr}{m}{n}\selectfont\color{color_29791}A tabela abaixo apresenta todas as instruções junto com os seus formatos }
\put(41.7,-111.3521){\fontsize{12}{1}\usefont{T1}{cmr}{m}{n}\selectfont\color{color_29791}correspondentes. }
\put(41.7,-137.1509){\fontsize{12}{1}\usefont{T1}{cmr}{m}{n}\selectfont\color{color_29791} }
\put(526.95,-429.1938){\fontsize{12}{1}\usefont{T1}{cmr}{m}{n}\selectfont\color{color_29791} }
\put(151.7167,-454.0547){\fontsize{11}{1}\usefont{T1}{cmr}{b}{n}\selectfont\color{color_29791}Figura 1. Tabela de Instruções do MIPS de 16 Bits. }
\put(41.7,-479.6416){\fontsize{12}{1}\usefont{T1}{cmr}{m}{n}\selectfont\color{color_29791} }
\put(41.7,-505.4404){\fontsize{12}{1}\usefont{T1}{cmr}{m}{n}\selectfont\color{color_29791}O MIPS de 16 bits mantém a filosofia de simplicidade e eficiência da arquitetura RISC, }
\put(41.7,-519.2393){\fontsize{12}{1}\usefont{T1}{cmr}{m}{n}\selectfont\color{color_29791}mas adaptado para um espaço de endereçamento e largura de barramento menores. As }
\put(41.7,-533.0381){\fontsize{12}{1}\usefont{T1}{cmr}{m}{n}\selectfont\color{color_29791}instruções são projetadas para realizar operações específicas de forma direta, }
\put(41.7,-546.8369){\fontsize{12}{1}\usefont{T1}{cmr}{m}{n}\selectfont\color{color_29791}favorecendo alta performance e simplicidade no hardware. Além disso, o conjunto de }
\put(41.7,-560.6357){\fontsize{12}{1}\usefont{T1}{cmr}{m}{n}\selectfont\color{color_29791}instruções é ortogonal, garantindo facilidade na programação e compilação, mesmo em }
\put(41.7,-574.4346){\fontsize{12}{1}\usefont{T1}{cmr}{m}{n}\selectfont\color{color_29791}um ambiente com recursos limitados. }
\put(41.7,-600.2334){\fontsize{12}{1}\usefont{T1}{cmr}{m}{n}\selectfont\color{color_29791} }
\put(41.7,-626.0322){\fontsize{12}{1}\usefont{T1}{cmr}{b}{n}\selectfont\color{color_29791}4. CIRCUITOS DESENVOLVIDOS }
\put(41.7,-651.8311){\fontsize{12}{1}\usefont{T1}{cmr}{m}{n}\selectfont\color{color_29791} A seguir, é apresentada uma descrição detalhada de cada componente }
\put(41.7,-665.6299){\fontsize{12}{1}\usefont{T1}{cmr}{m}{n}\selectfont\color{color_29791}implementado}
\put(43.20001,-429.1938){\includegraphics[width=482.25pt,height=276pt]{latexImage_2302fa9ee242855524fe33ba536ab935.png}}
\end{picture}
\newpage
\begin{tikzpicture}[overlay]
\path(0pt,0pt);
\filldraw[color_283006][nonzero rule]
(-15pt, 10pt) -- (580.5pt, 10pt)
 -- (580.5pt, 10pt)
 -- (580.5pt, -832.25pt)
 -- (580.5pt, -832.25pt)
 -- (-15pt, -832.25pt) -- cycle
;
\end{tikzpicture}
\begin{picture}(-5,0)(2.5,0)
\put(526.95,-326.4492){\fontsize{12}{1}\usefont{T1}{cmr}{m}{n}\selectfont\color{color_29791} }
\put(209.0363,-351.3101){\fontsize{11}{1}\usefont{T1}{cmr}{b}{n}\selectfont\color{color_29791}Figura 2. Esquema do MIPS. }
\put(41.7,-376.8975){\fontsize{12}{1}\usefont{T1}{cmr}{b}{n}\selectfont\color{color_29791}4.1. Somador de 6 Bits }
\put(41.7,-391.7583){\fontsize{11}{1}\usefont{T1}{cmr}{m}{n}\selectfont\color{color_29791} }
\put(41.7,-407.3457){\fontsize{11}{1}\usefont{T1}{cmr}{m}{n}\selectfont\color{color_29791} O Somador de 6 Bits desenvolvido no Logisim, conforme mostrado no circuito, é }
\put(41.7,-421.1445){\fontsize{12}{1}\usefont{T1}{cmr}{m}{n}\selectfont\color{color_29791}composto por seis módulos de somadores completos (full adders) conectados em cascata }
\put(41.7,-434.9434){\fontsize{12}{1}\usefont{T1}{cmr}{m}{n}\selectfont\color{color_29791}para realizar a adição de dois números binários de 6 bits, representados pelas entradas A }
\put(41.7,-448.7422){\fontsize{12}{1}\usefont{T1}{cmr}{m}{n}\selectfont\color{color_29791}e B. Cada somador completo processa um bit correspondente de A e B, considerando }
\put(41.7,-462.541){\fontsize{12}{1}\usefont{T1}{cmr}{m}{n}\selectfont\color{color_29791}também o bit de transporte (carry-in) da operação do bit menos significativo para o mais }
\put(41.7,-476.3398){\fontsize{12}{1}\usefont{T1}{cmr}{m}{n}\selectfont\color{color_29791}significativo. No circuito, cada somador completo possui três entradas: A, B e Cin }
\put(41.7,-490.1387){\fontsize{12}{1}\usefont{T1}{cmr}{m}{n}\selectfont\color{color_29791}(carry-in), e gera duas saídas: S (soma) e Cout (carry-out). O primeiro somador, }
\put(41.7,-503.9375){\fontsize{12}{1}\usefont{T1}{cmr}{m}{n}\selectfont\color{color_29791}correspondente ao bit menos significativo, recebe o carry-in inicial como 0, enquanto os }
\put(41.7,-517.7363){\fontsize{12}{1}\usefont{T1}{cmr}{m}{n}\selectfont\color{color_29791}somadores subsequentes utilizam o carry-out do estágio anterior como carry-in. As saídas }
\put(41.7,-531.5352){\fontsize{12}{1}\usefont{T1}{cmr}{m}{n}\selectfont\color{color_29791}de soma de todos os somadores são conectadas em série para formar a saída final de 6 }
\put(41.7,-545.334){\fontsize{12}{1}\usefont{T1}{cmr}{m}{n}\selectfont\color{color_29791}bits (Saída), que representa o resultado da adição. O funcionamento do circuito é simples }
\put(41.7,-559.1328){\fontsize{12}{1}\usefont{T1}{cmr}{m}{n}\selectfont\color{color_29791}e eficiente: cada estágio do somador calcula a soma parcial e propaga o carry para o }
\put(41.7,-572.9316){\fontsize{12}{1}\usefont{T1}{cmr}{m}{n}\selectfont\color{color_29791}próximo estágio. Esse método permite que o circuito adicione qualquer par de números }
\put(41.7,-586.7305){\fontsize{12}{1}\usefont{T1}{cmr}{m}{n}\selectfont\color{color_29791}binários}
\put(43.20001,-326.4492){\includegraphics[width=482.25pt,height=278.25pt]{latexImage_9671ac93a18f131eb165e5cfda686410.png}}
\end{picture}
\newpage
\begin{tikzpicture}[overlay]
\path(0pt,0pt);
\filldraw[color_283006][nonzero rule]
(-15pt, 10pt) -- (580.5pt, 10pt)
 -- (580.5pt, 10pt)
 -- (580.5pt, -832.25pt)
 -- (580.5pt, -832.25pt)
 -- (-15pt, -832.25pt) -- cycle
;
\end{tikzpicture}
\begin{picture}(-5,0)(2.5,0)
\put(463.3878,-294.1992){\fontsize{11}{1}\usefont{T1}{cmr}{m}{n}\selectfont\color{color_29791} }
\put(177.8927,-308.8491){\fontsize{11}{1}\usefont{T1}{cmr}{b}{n}\selectfont\color{color_29791}Figura 3. Circuito do Somador de 6 Bits. }
\put(282.6378,-323.4976){\fontsize{11}{1}\usefont{T1}{cmr}{b}{n}\selectfont\color{color_29791} }
\put(41.7,-349.085){\fontsize{12}{1}\usefont{T1}{cmr}{b}{n}\selectfont\color{color_29791}4.2. Contador de Programa }
\put(41.7,-363.9458){\fontsize{11}{1}\usefont{T1}{cmr}{m}{n}\selectfont\color{color_29791} }
\put(41.7,-379.5332){\fontsize{11}{1}\usefont{T1}{cmr}{m}{n}\selectfont\color{color_29791} O Contador de Programa (PC), representado no circuito, é responsável por }
\put(41.7,-393.332){\fontsize{12}{1}\usefont{T1}{cmr}{m}{n}\selectfont\color{color_29791}controlar a sequência de execução das instruções no processador MIPS. Ele armazena o }
\put(41.7,-407.1309){\fontsize{12}{1}\usefont{T1}{cmr}{m}{n}\selectfont\color{color_29791}endereço da próxima instrução a ser buscada na memória e é projetado para incrementar }
\put(41.7,-420.9297){\fontsize{12}{1}\usefont{T1}{cmr}{m}{n}\selectfont\color{color_29791}automaticamente a cada ciclo de clock, permitindo o fluxo sequencial do programa. O }
\put(41.7,-434.7285){\fontsize{12}{1}\usefont{T1}{cmr}{m}{n}\selectfont\color{color_29791}circuito é composto por registradores tipo D flip-flop, que armazenam os bits do endereço }
\put(41.7,-448.5273){\fontsize{12}{1}\usefont{T1}{cmr}{m}{n}\selectfont\color{color_29791}atual. A entrada CLK (clock) controla a sincronização do incremento, enquanto a entrada }
\put(41.7,-462.3262){\fontsize{12}{1}\usefont{T1}{cmr}{m}{n}\selectfont\color{color_29791}CLR (clear) permite zerar o valor do contador, reiniciando sua operação. Adicionalmente, }
\put(41.7,-476.125){\fontsize{12}{1}\usefont{T1}{cmr}{m}{n}\selectfont\color{color_29791}há uma entrada de controle chamada “PC\_On”, que habilita ou desabilita o funcionamento }
\put(41.7,-489.9238){\fontsize{12}{1}\usefont{T1}{cmr}{m}{n}\selectfont\color{color_29791}do contador, e a entrada A, que possibilita carregar manualmente um endereço específico }
\put(41.7,-503.7227){\fontsize{12}{1}\usefont{T1}{cmr}{m}{n}\selectfont\color{color_29791}no contador, permitindo saltos ou alterações no fluxo do programa. O funcionamento do }
\put(41.7,-517.5215){\fontsize{12}{1}\usefont{T1}{cmr}{m}{n}\selectfont\color{color_29791}circuito se dá em dois modos principais: no modo sequencial, o contador incrementa }
\put(41.7,-531.3203){\fontsize{12}{1}\usefont{T1}{cmr}{m}{n}\selectfont\color{color_29791}automaticamente o valor armazenado a cada ciclo de clock, enquanto no modo de }
\put(41.7,-545.1191){\fontsize{12}{1}\usefont{T1}{cmr}{m}{n}\selectfont\color{color_29791}carregamento manual, o valor presente na entrada A é transferido para os flip-flops, }
\put(41.7,-558.918){\fontsize{12}{1}\usefont{T1}{cmr}{m}{n}\selectfont\color{color_29791}alterando diretamente o endereço armazenado. A saída do contador é conectada a outros }
\put(41.7,-572.7168){\fontsize{12}{1}\usefont{T1}{cmr}{m}{n}\selectfont\color{color_29791}componentes do processador, como a unidade de controle e a memória, determinando }
\put(41.7,-586.5156){\fontsize{12}{1}\usefont{T1}{cmr}{m}{n}\selectfont\color{color_29791}qual instrução será executada em seguida. Esse circuito é fundamental para a execução }
\put(41.7,-600.3145){\fontsize{12}{1}\usefont{T1}{cmr}{m}{n}\selectfont\color{color_29791}ordenada de instruções no processador, garantindo o controle preciso do fluxo de }
\put(41.7,-614.1133){\fontsize{12}{1}\usefont{T1}{cmr}{m}{n}\selectfont\color{color_29791}execução, seja de forma sequencial ou com alterações controladas, como em instruções }
\put(41.7,-627.9121){\fontsize{12}{1}\usefont{T1}{cmr}{m}{n}\selectfont\color{color_29791}de}
\put(103.3878,-294.1992){\includegraphics[width=358.5pt,height=246pt]{latexImage_5bd5f2300971fff9ccc8bb1da442b5e9.png}}
\end{picture}
\newpage
\begin{tikzpicture}[overlay]
\path(0pt,0pt);
\filldraw[color_283006][nonzero rule]
(-15pt, 10pt) -- (580.5pt, 10pt)
 -- (580.5pt, 10pt)
 -- (580.5pt, -832.25pt)
 -- (580.5pt, -832.25pt)
 -- (-15pt, -832.25pt) -- cycle
;
\end{tikzpicture}
\begin{picture}(-5,0)(2.5,0)
\put(465.2628,-253.6992){\fontsize{11}{1}\usefont{T1}{cmr}{m}{n}\selectfont\color{color_29791} }
\put(166.5972,-268.3491){\fontsize{11}{1}\usefont{T1}{cmr}{b}{n}\selectfont\color{color_29791}Figura 4. Circuito do Contador de Programa. }
\put(282.6378,-282.9976){\fontsize{11}{1}\usefont{T1}{cmr}{b}{n}\selectfont\color{color_29791} }
\put(41.7,-308.585){\fontsize{12}{1}\usefont{T1}{cmr}{b}{n}\selectfont\color{color_29791}4.3. Memória de Instruções (ROM) }
\put(41.7,-323.4458){\fontsize{11}{1}\usefont{T1}{cmr}{m}{n}\selectfont\color{color_29791} }
\put(41.7,-339.0332){\fontsize{11}{1}\usefont{T1}{cmr}{m}{n}\selectfont\color{color_29791} A Memória de Instruções, implementada no circuito utilizando uma memória ROM }
\put(41.7,-352.832){\fontsize{12}{1}\usefont{T1}{cmr}{m}{n}\selectfont\color{color_29791}(Read-Only Memory), é responsável por armazenar o conjunto de instruções que serão }
\put(41.7,-366.6309){\fontsize{12}{1}\usefont{T1}{cmr}{m}{n}\selectfont\color{color_29791}executadas pelo processador MIPS. Essa memória é organizada em 64 palavras de 16 }
\put(41.7,-380.4297){\fontsize{12}{1}\usefont{T1}{cmr}{m}{n}\selectfont\color{color_29791}bits, conforme exibido no circuito, e cada endereço armazena uma instrução codificada }
\put(41.7,-394.2285){\fontsize{12}{1}\usefont{T1}{cmr}{m}{n}\selectfont\color{color_29791}em HEX. A entrada de 5 bits no barramento de endereços permite acessar qualquer uma }
\put(41.7,-408.0273){\fontsize{12}{1}\usefont{T1}{cmr}{m}{n}\selectfont\color{color_29791}das 64 posições de memória, enquanto a saída de 16 bits fornece a instrução }
\put(41.7,-421.8262){\fontsize{12}{1}\usefont{T1}{cmr}{m}{n}\selectfont\color{color_29791}correspondente ao endereço selecionado. A ROM, por ser uma memória de leitura, }
\put(41.7,-435.625){\fontsize{12}{1}\usefont{T1}{cmr}{m}{n}\selectfont\color{color_29791}contém um conjunto de instruções previamente gravadas e não pode ser alterada durante }
\put(41.7,-449.4238){\fontsize{12}{1}\usefont{T1}{cmr}{m}{n}\selectfont\color{color_29791}a execução, o que a torna ideal para a simulação de programas fixos. No contexto do }
\put(41.7,-463.2227){\fontsize{12}{1}\usefont{T1}{cmr}{m}{n}\selectfont\color{color_29791}funcionamento do processador, o Contador de Programa (PC) fornece o endereço da }
\put(41.7,-477.0215){\fontsize{12}{1}\usefont{T1}{cmr}{m}{n}\selectfont\color{color_29791}próxima instrução, que é utilizado para acessar a ROM e carregar a instrução na saída. }
\put(41.7,-490.8203){\fontsize{12}{1}\usefont{T1}{cmr}{m}{n}\selectfont\color{color_29791}Essa instrução é então decodificada e executada pelos demais componentes do }
\put(41.7,-504.6191){\fontsize{12}{1}\usefont{T1}{cmr}{m}{n}\selectfont\color{color_29791}processador. A utilização da ROM no Logisim para simular a Memória de Instruções }
\put(41.7,-518.418){\fontsize{12}{1}\usefont{T1}{cmr}{m}{n}\selectfont\color{color_29791}simplifica o desenvolvimento e a execução de programas no processador MIPS, além de }
\put(41.7,-532.2168){\fontsize{12}{1}\usefont{T1}{cmr}{m}{n}\selectfont\color{color_29791}permitir a visualização e manipulação das instruções diretamente. Esse componente é }
\put(41.7,-546.0156){\fontsize{12}{1}\usefont{T1}{cmr}{m}{n}\selectfont\color{color_29791}essencial para garantir o funcionamento correto do ciclo de busca e execução no }
\put(41.7,-559.8145){\fontsize{12}{1}\usefont{T1}{cmr}{m}{n}\selectfont\color{color_29791}processador. }
\put(41.7,-575.6123){\fontsize{12}{1}\usefont{T1}{cmr}{m}{n}\selectfont\color{color_29791} }
\put(336.6378,-750.4053){\fontsize{11}{1}\usefont{T1}{cmr}{m}{n}\selectfont\color{color_29791} }
\put(182.7688,-765.0542){\fontsize{11}{1}\usefont{T1}{cmr}{b}{n}\selectfont\color{color_29791}F}
\put(101.5128,-253.6992){\includegraphics[width=362.25pt,height=205.5pt]{latexImage_56f9633ebfa4c3240f055c39d70316f4.png}}
\put(230.1378,-750.4053){\includegraphics[width=105pt,height=168.75pt]{latexImage_751e6036b6c1a40351c1181be03c4177.png}}
\end{picture}
\newpage
\begin{tikzpicture}[overlay]
\path(0pt,0pt);
\filldraw[color_283006][nonzero rule]
(-15pt, 10pt) -- (580.5pt, 10pt)
 -- (580.5pt, 10pt)
 -- (580.5pt, -832.25pt)
 -- (580.5pt, -832.25pt)
 -- (-15pt, -832.25pt) -- cycle
;
\end{tikzpicture}
\begin{picture}(-5,0)(2.5,0)
\put(41.7,-57.95508){\fontsize{12}{1}\usefont{T1}{cmr}{b}{n}\selectfont\color{color_29791}4.4. Banco de Registradores (8x16) }
\put(41.7,-72.81689){\fontsize{11}{1}\usefont{T1}{cmr}{m}{n}\selectfont\color{color_29791} }
\put(41.7,-88.4043){\fontsize{11}{1}\usefont{T1}{cmr}{m}{n}\selectfont\color{color_29791} O banco de registradores é um componente fundamental em arquiteturas como a }
\put(41.7,-102.2031){\fontsize{12}{1}\usefont{T1}{cmr}{m}{n}\selectfont\color{color_29791}do processador MIPS, sendo responsável por armazenar temporariamente dados }
\put(41.7,-116.002){\fontsize{12}{1}\usefont{T1}{cmr}{m}{n}\selectfont\color{color_29791}utilizados nas operações de processamento. No circuito apresentado, temos um banco de }
\put(41.7,-129.8008){\fontsize{12}{1}\usefont{T1}{cmr}{m}{n}\selectfont\color{color_29791}registradores 8x16, que consiste em 8 registradores, cada um com 16 bits de largura. A }
\put(41.7,-143.5996){\fontsize{12}{1}\usefont{T1}{cmr}{m}{n}\selectfont\color{color_29791}entrada principal, indicada como "Entrada", é o barramento de dados que alimenta os }
\put(41.7,-157.3984){\fontsize{12}{1}\usefont{T1}{cmr}{m}{n}\selectfont\color{color_29791}registradores. A seleção de qual registrador será escrito é realizada por um }
\put(41.7,-171.1973){\fontsize{12}{1}\usefont{T1}{cmr}{m}{n}\selectfont\color{color_29791}demultiplexador (DMX), controlado pelo sinal "Wreg". Esse sinal, juntamente com o pulso }
\put(41.7,-184.9961){\fontsize{12}{1}\usefont{T1}{cmr}{m}{n}\selectfont\color{color_29791}de "Write", determina o momento e o registrador específico para realizar a escrita de }
\put(41.7,-198.7949){\fontsize{12}{1}\usefont{T1}{cmr}{m}{n}\selectfont\color{color_29791}dados. Adicionalmente, há um sinal de "Reset" que possibilita a reinicialização de todos }
\put(41.7,-212.5938){\fontsize{12}{1}\usefont{T1}{cmr}{m}{n}\selectfont\color{color_29791}os registradores para um estado inicial, tipicamente zerado. Cada registrador possui uma }
\put(41.7,-226.3926){\fontsize{12}{1}\usefont{T1}{cmr}{m}{n}\selectfont\color{color_29791}entrada de habilitação (WE - Write Enable), permitindo que apenas o registrador }
\put(41.7,-240.1914){\fontsize{12}{1}\usefont{T1}{cmr}{m}{n}\selectfont\color{color_29791}selecionado receba os dados da entrada. Durante o processo de leitura, dois }
\put(41.7,-253.9902){\fontsize{12}{1}\usefont{T1}{cmr}{m}{n}\selectfont\color{color_29791}multiplexadores (MUX) selecionam os registradores que terão seus valores encaminhados }
\put(41.7,-267.7891){\fontsize{12}{1}\usefont{T1}{cmr}{m}{n}\selectfont\color{color_29791}para as saídas "Rdata1" e "Rdata2", correspondendo às saídas "Sr1" e "Sr2", }
\put(41.7,-281.5879){\fontsize{12}{1}\usefont{T1}{cmr}{m}{n}\selectfont\color{color_29791}respectivamente. Os sinais de controle dos multiplexadores determinam quais }
\put(41.7,-295.3867){\fontsize{12}{1}\usefont{T1}{cmr}{m}{n}\selectfont\color{color_29791}registradores serão lidos e enviados às respectivas saídas, permitindo a utilização dos }
\put(41.7,-309.1855){\fontsize{12}{1}\usefont{T1}{cmr}{m}{n}\selectfont\color{color_29791}dados armazenados no banco em outras partes do processador. Este circuito exemplifica }
\put(41.7,-322.9844){\fontsize{12}{1}\usefont{T1}{cmr}{m}{n}\selectfont\color{color_29791}um banco de registradores típico utilizado no processador MIPS, oferecendo operações }
\put(41.7,-336.7832){\fontsize{12}{1}\usefont{T1}{cmr}{m}{n}\selectfont\color{color_29791}básicas de leitura e escrita com controle eficiente, enquanto garante a capacidade de }
\put(41.7,-350.582){\fontsize{12}{1}\usefont{T1}{cmr}{m}{n}\selectfont\color{color_29791}armazenamento e acesso rápido aos dados necessários para a execução de instruções. }
\put(41.7,-366.3799){\fontsize{12}{1}\usefont{T1}{cmr}{m}{n}\selectfont\color{color_29791} }
\put(526.95,-509.6729){\fontsize{11}{1}\usefont{T1}{cmr}{m}{n}\selectfont\color{color_29791} }
\put(162.9233,-524.3218){\fontsize{11}{1}\usefont{T1}{cmr}{b}{n}\selectfont\color{color_29791}Figura 6. Circuito do Banco de Registradores. }
\put(282.6378,-538.9712){\fontsize{11}{1}\usefont{T1}{cmr}{b}{n}\selectfont\color{color_29791} }
\put(41.7,-564.5586){\fontsize{12}{1}\usefont{T1}{cmr}{b}{n}\selectfont\color{color_29791}4.5. Unidade Lógica e Aritmética (ULA) }
\put(41.7,-579.4194){\fontsize{11}{1}\usefont{T1}{cmr}{m}{n}\selectfont\color{color_29791} }
\put(41.7,-595.0059){\fontsize{11}{1}\usefont{T1}{cmr}{m}{n}\selectfont\color{color_29791} A Unidade Lógica e Aritmética (ULA) é um componente essencial de }
\put(41.7,-608.8047){\fontsize{12}{1}\usefont{T1}{cmr}{m}{n}\selectfont\color{color_29791}processadores como o MIPS, sendo responsável pela execução de operações aritméticas }
\put(41.7,-622.6035){\fontsize{12}{1}\usefont{T1}{cmr}{m}{n}\selectfont\color{color_29791}e lógicas definidas pelas instruções do programa. O circuito apresentado representa uma }
\put(41.7,-636.4023){\fontsize{12}{1}\usefont{T1}{cmr}{m}{n}\selectfont\color{color_29791}ULA configurada para operar com dados de 16 bits, recebendo duas entradas principais, }
\put(41.7,-650.2012){\fontsize{12}{1}\usefont{T1}{cmr}{m}{n}\selectfont\color{color_29791}"A" e "B", e um sinal de controle denominado "Selector", que especifica a operação a ser }
\put(41.7,-664){\fontsize{12}{1}\usefont{T1}{cmr}{m}{n}\selectfont\color{color_29791}realizada. O seletor, com 3 bits de largura, controla qual operação será executada, }
\put(41.7,-677.7988){\fontsize{12}{1}\usefont{T1}{cmr}{m}{n}\selectfont\color{color_29791}conforme descrito na tabela de mnemônicos apresentada no circuito. Entre as operações }
\put(41.7,-691.5977){\fontsize{12}{1}\usefont{T1}{cmr}{m}{n}\selectfont\color{color_29791}suportadas estão soma (SOMA), subtração (SUB), operações lógicas como AND, OR, }
\put(41.7,-705.3965){\fontsize{12}{1}\usefont{T1}{cmr}{m}{n}\selectfont\color{color_29791}NAND, NOR e XOR. A execução das operações é feita por uma combinação de portas }
\put(41.7,-719.1953){\fontsize{12}{1}\usefont{T1}{cmr}{m}{n}\selectfont\color{color_29791}lógicas e um somador/subtrator. O somador/subtrator é ativado para realizar cálculos }
\put(41.7,-732.9941){\fontsize{12}{1}\usefont{T1}{cmr}{m}{n}\selectfont\color{color_29791}aritméticos, dependendo do valor do seletor. Operações lógicas, por sua vez, são }
\put(41.7,-746.793){\fontsize{12}{1}\usefont{T1}{cmr}{m}{n}\selectfont\color{color_29791}implementadas diretamente utilizando portas AND, OR, NAND, NOR e XOR, que }
\put(41.7,-760.5918){\fontsize{12}{1}\usefont{T1}{cmr}{m}{n}\selectfont\color{color_29791}processam}
\put(43.20001,-509.6729){\includegraphics[width=482.25pt,height=137.25pt]{latexImage_82de56b64bcc0dab16cf9d2c6c393647.png}}
\end{picture}
\newpage
\begin{tikzpicture}[overlay]
\path(0pt,0pt);
\filldraw[color_283006][nonzero rule]
(-15pt, 10pt) -- (580.5pt, 10pt)
 -- (580.5pt, 10pt)
 -- (580.5pt, -832.25pt)
 -- (580.5pt, -832.25pt)
 -- (-15pt, -832.25pt) -- cycle
;
\end{tikzpicture}
\begin{picture}(-5,0)(2.5,0)
\put(41.7,-57.95508){\fontsize{12}{1}\usefont{T1}{cmr}{m}{n}\selectfont\color{color_29791}resultados das operações são encaminhados para um multiplexador (MUX). Este }
\put(41.7,-71.75391){\fontsize{12}{1}\usefont{T1}{cmr}{m}{n}\selectfont\color{color_29791}multiplexador seleciona qual resultado será enviado à saída "Saída", de acordo com o }
\put(41.7,-85.55273){\fontsize{12}{1}\usefont{T1}{cmr}{m}{n}\selectfont\color{color_29791}valor do seletor, garantindo que o resultado correto seja utilizado para os cálculos }
\put(41.7,-99.35156){\fontsize{12}{1}\usefont{T1}{cmr}{m}{n}\selectfont\color{color_29791}subsequentes ou armazenado no banco de registradores. Essa implementação da ULA é }
\put(41.7,-113.1504){\fontsize{12}{1}\usefont{T1}{cmr}{m}{n}\selectfont\color{color_29791}um exemplo claro de como operações aritméticas e lógicas são realizadas em um }
\put(41.7,-126.9492){\fontsize{12}{1}\usefont{T1}{cmr}{m}{n}\selectfont\color{color_29791}processador. O circuito destaca a interação entre os sinais de controle e o hardware }
\put(41.7,-140.748){\fontsize{12}{1}\usefont{T1}{cmr}{m}{n}\selectfont\color{color_29791}subjacente para fornecer uma unidade funcional que atende às necessidades de }
\put(41.7,-154.5469){\fontsize{12}{1}\usefont{T1}{cmr}{m}{n}\selectfont\color{color_29791}execução de instruções no MIPS. }
\put(507.2628,-474.8408){\fontsize{12}{1}\usefont{T1}{cmr}{m}{n}\selectfont\color{color_29791} }
\put(282.6378,-489.7017){\fontsize{11}{1}\usefont{T1}{cmr}{m}{n}\selectfont\color{color_29791} }
\put(215.1523,-504.3501){\fontsize{11}{1}\usefont{T1}{cmr}{b}{n}\selectfont\color{color_29791}Figura 7. Circuito da ULA. }
\put(41.7,-528.9995){\fontsize{11}{1}\usefont{T1}{cmr}{b}{n}\selectfont\color{color_29791} }
\put(41.7,-554.5859){\fontsize{12}{1}\usefont{T1}{cmr}{b}{n}\selectfont\color{color_29791}4.6. Memória de Dados (RAM) }
\put(41.7,-569.4478){\fontsize{11}{1}\usefont{T1}{cmr}{m}{n}\selectfont\color{color_29791} }
\put(41.7,-585.0342){\fontsize{11}{1}\usefont{T1}{cmr}{m}{n}\selectfont\color{color_29791} A memória de dados (RAM) é utilizada para armazenar temporariamente }
\put(41.7,-598.833){\fontsize{12}{1}\usefont{T1}{cmr}{m}{n}\selectfont\color{color_29791}informações necessárias durante a execução do processador, como dados intermediários }
\put(41.7,-612.6318){\fontsize{12}{1}\usefont{T1}{cmr}{m}{n}\selectfont\color{color_29791}ou valores de instruções. No Logisim, a RAM é configurável, permitindo definir o número }
\put(41.7,-626.4307){\fontsize{12}{1}\usefont{T1}{cmr}{m}{n}\selectfont\color{color_29791}de endereços e a largura de cada palavra. Ela possui como entradas o endereço, os }
\put(41.7,-640.2295){\fontsize{12}{1}\usefont{T1}{cmr}{m}{n}\selectfont\color{color_29791}dados a serem armazenados e sinais de controle, como o Write Enable (WE), que habilita }
\put(41.7,-654.0283){\fontsize{12}{1}\usefont{T1}{cmr}{m}{n}\selectfont\color{color_29791}a escrita, e o Clock (CLK), que sincroniza as operações. Para leitura, o endereço }
\put(41.7,-667.8271){\fontsize{12}{1}\usefont{T1}{cmr}{m}{n}\selectfont\color{color_29791}especifica a posição da memória, e o dado armazenado é retornado na saída. Para }
\put(41.7,-681.626){\fontsize{12}{1}\usefont{T1}{cmr}{m}{n}\selectfont\color{color_29791}escrita, os dados de entrada são armazenados no endereço indicado quando o sinal de }
\put(41.7,-695.4248){\fontsize{12}{1}\usefont{T1}{cmr}{m}{n}\selectfont\color{color_29791}escrita está ativo. A RAM do Logisim facilita a simulação e integração com outros }
\put(41.7,-709.2236){\fontsize{12}{1}\usefont{T1}{cmr}{m}{n}\selectfont\color{color_29791}componentes, como a ULA e o banco de registradores, refletindo o funcionamento real de }
\put(41.7,-723.0225){\fontsize{12}{1}\usefont{T1}{cmr}{m}{n}\selectfont\color{color_29791}um}
\put(59.51279,-474.8418){\includegraphics[width=446.25pt,height=314.25pt]{latexImage_f5c46570a2dc63e5ede85c5d3262fd7d.png}}
\end{picture}
\newpage
\begin{tikzpicture}[overlay]
\path(0pt,0pt);
\filldraw[color_283006][nonzero rule]
(-15pt, 10pt) -- (580.5pt, 10pt)
 -- (580.5pt, 10pt)
 -- (580.5pt, -832.25pt)
 -- (580.5pt, -832.25pt)
 -- (-15pt, -832.25pt) -- cycle
;
\end{tikzpicture}
\begin{picture}(-5,0)(2.5,0)
\put(372.6378,-321.1992){\fontsize{11}{1}\usefont{T1}{cmr}{m}{n}\selectfont\color{color_29791} }
\put(183.0748,-335.8491){\fontsize{11}{1}\usefont{T1}{cmr}{b}{n}\selectfont\color{color_29791}Figura 8. Imagem da RAM do Sistema. }
\put(282.6378,-350.4976){\fontsize{11}{1}\usefont{T1}{cmr}{b}{n}\selectfont\color{color_29791} }
\put(41.7,-366.0859){\fontsize{12}{1}\usefont{T1}{cmr}{b}{n}\selectfont\color{color_29791}4.7 Controlador da ULA }
\put(41.7,-381.8838){\fontsize{12}{1}\usefont{T1}{cmr}{b}{n}\selectfont\color{color_29791} }
\put(41.7,-397.6816){\fontsize{12}{1}\usefont{T1}{cmr}{b}{n}\selectfont\color{color_29791} O Controlador da ULA (ALUController) é um componente essencial no }
\put(41.7,-411.4805){\fontsize{12}{1}\usefont{T1}{cmr}{m}{n}\selectfont\color{color_29791}microprocessador MIPS, responsável por interpretar os sinais de controle e definir as }
\put(41.7,-425.2793){\fontsize{12}{1}\usefont{T1}{cmr}{m}{n}\selectfont\color{color_29791}operações que a Unidade Lógica e Aritmética (ULA) deve executar. No circuito }
\put(41.7,-439.0781){\fontsize{12}{1}\usefont{T1}{cmr}{m}{n}\selectfont\color{color_29791}apresentado, o Controlador da ULA recebe entradas como o campo funct das instruções }
\put(41.7,-452.877){\fontsize{12}{1}\usefont{T1}{cmr}{m}{n}\selectfont\color{color_29791}do tipo R e o opcode das instruções do tipo I, gerando os sinais de controle necessários }
\put(41.7,-466.6758){\fontsize{12}{1}\usefont{T1}{cmr}{m}{n}\selectfont\color{color_29791}para configurar a ULA. Esses sinais, representados pelo OpALU, determinam a operação }
\put(41.7,-480.4746){\fontsize{12}{1}\usefont{T1}{cmr}{m}{n}\selectfont\color{color_29791}específica a ser realizada, como operações aritméticas, lógicas ou manipulações de }
\put(41.7,-494.2734){\fontsize{12}{1}\usefont{T1}{cmr}{m}{n}\selectfont\color{color_29791}dados. Por exemplo, o valor 011 corresponde a uma operação do tipo R, enquanto 100 }
\put(41.7,-508.0723){\fontsize{12}{1}\usefont{T1}{cmr}{m}{n}\selectfont\color{color_29791}indica uma operação de adição imediata (ADDI). O circuito suporta uma variedade de }
\put(41.7,-521.8711){\fontsize{12}{1}\usefont{T1}{cmr}{m}{n}\selectfont\color{color_29791}operações, incluindo ADDI, SUBI, ORI, ANDI e instruções do tipo R, além de um estado }
\put(41.7,-535.6699){\fontsize{12}{1}\usefont{T1}{cmr}{m}{n}\selectfont\color{color_29791}"NADA" para operações que não requerem ação da ULA. O Controlador da ULA utiliza }
\put(41.7,-549.4688){\fontsize{12}{1}\usefont{T1}{cmr}{m}{n}\selectfont\color{color_29791}uma lógica combinacional para mapear as entradas (funct e opcode) nos sinais de }
\put(41.7,-563.2676){\fontsize{12}{1}\usefont{T1}{cmr}{m}{n}\selectfont\color{color_29791}controle apropriados. Essa lógica é implementada por meio de portas lógicas que }
\put(41.7,-577.0664){\fontsize{12}{1}\usefont{T1}{cmr}{m}{n}\selectfont\color{color_29791}traduzem os valores de entrada em sinais de saída (OpALU). O circuito também inclui um }
\put(41.7,-590.8652){\fontsize{12}{1}\usefont{T1}{cmr}{m}{n}\selectfont\color{color_29791}funct pool, que é utilizado para identificar operações específicas dentro das instruções do }
\put(41.7,-604.6641){\fontsize{12}{1}\usefont{T1}{cmr}{m}{n}\selectfont\color{color_29791}tipo R. O sinal OpALU gerado pelo Controlador é enviado diretamente para a ULA, que }
\put(41.7,-618.4629){\fontsize{12}{1}\usefont{T1}{cmr}{m}{n}\selectfont\color{color_29791}executa a operação correspondente com base nos operandos fornecidos, garantindo que }
\put(41.7,-632.2617){\fontsize{12}{1}\usefont{T1}{cmr}{m}{n}\selectfont\color{color_29791}as operações aritméticas e lógicas sejam realizadas de maneira precisa e eficiente. Em }
\put(41.7,-646.0605){\fontsize{12}{1}\usefont{T1}{cmr}{m}{n}\selectfont\color{color_29791}resumo, o Controlador da ULA desempenha um papel fundamental na execução das }
\put(41.7,-659.8594){\fontsize{12}{1}\usefont{T1}{cmr}{m}{n}\selectfont\color{color_29791}instruções do microprocessador MIPS. Ele traduz os sinais de alto nível em comandos de }
\put(41.7,-673.6582){\fontsize{12}{1}\usefont{T1}{cmr}{m}{n}\selectfont\color{color_29791}baixo nível que a ULA pode interpretar, garantindo que as operações sejam realizadas }
\put(41.7,-687.457){\fontsize{12}{1}\usefont{T1}{cmr}{m}{n}\selectfont\color{color_29791}corretamente. Sua implementação combina lógica combinacional e mapeamento de }
\put(41.7,-701.2559){\fontsize{12}{1}\usefont{T1}{cmr}{m}{n}\selectfont\color{color_29791}operações, permitindo que o processador execute uma variedade de instruções de forma }
\put(41.7,-715.0547){\fontsize{12}{1}\usefont{T1}{cmr}{m}{n}\selectfont\color{color_29791}eficiente e precisa. Esse componente é essencial para a modularidade e o desempenho }
\put(41.7,-728.8535){\fontsize{12}{1}\usefont{T1}{cmr}{m}{n}\selectfont\color{color_29791}da}
\put(194.1378,-321.1992){\includegraphics[width=177pt,height=273pt]{latexImage_9e5999895a9f422d44d261fe123d0eae.png}}
\end{picture}
\newpage
\begin{tikzpicture}[overlay]
\path(0pt,0pt);
\filldraw[color_283006][nonzero rule]
(-15pt, 10pt) -- (580.5pt, 10pt)
 -- (580.5pt, 10pt)
 -- (580.5pt, -832.25pt)
 -- (580.5pt, -832.25pt)
 -- (-15pt, -832.25pt) -- cycle
;
\end{tikzpicture}
\begin{picture}(-5,0)(2.5,0)
\put(485.7,-339.1992){\fontsize{12}{1}\usefont{T1}{cmr}{m}{n}\selectfont\color{color_29791} }
\put(173.9289,-354.062){\fontsize{11}{1}\usefont{T1}{cmr}{b}{n}\selectfont\color{color_29791}Figura 9. Imagem do Controlador da ULA. }
\put(282.6378,-368.7104){\fontsize{11}{1}\usefont{T1}{cmr}{b}{n}\selectfont\color{color_29791} }
\put(41.7,-384.2969){\fontsize{12}{1}\usefont{T1}{cmr}{b}{n}\selectfont\color{color_29791}4.8 Controlador de Sinais }
\put(41.7,-400.0957){\fontsize{12}{1}\usefont{T1}{cmr}{b}{n}\selectfont\color{color_29791} }
\put(41.7,-415.8945){\fontsize{12}{1}\usefont{T1}{cmr}{m}{n}\selectfont\color{color_29791} O Controlador de Sinais é um dos principais componentes do processador MIPS de }
\put(41.7,-429.6934){\fontsize{12}{1}\usefont{T1}{cmr}{m}{n}\selectfont\color{color_29791}16 bits, responsável por gerar os sinais de controle que coordenam a execução das }
\put(41.7,-443.4922){\fontsize{12}{1}\usefont{T1}{cmr}{m}{n}\selectfont\color{color_29791}instruções. Ele recebe como entrada o campo Opcode da instrução e, com base nesse }
\put(41.7,-457.291){\fontsize{12}{1}\usefont{T1}{cmr}{m}{n}\selectfont\color{color_29791}valor, ativa ou desativa sinais que controlam o funcionamento dos principais blocos do }
\put(41.7,-471.0898){\fontsize{12}{1}\usefont{T1}{cmr}{m}{n}\selectfont\color{color_29791}processador, como a ULA, os registradores e a memória. No circuito apresentado, o }
\put(41.7,-484.8887){\fontsize{12}{1}\usefont{T1}{cmr}{m}{n}\selectfont\color{color_29791}Opcode é utilizado para definir a configuração correta dos sinais de controle por meio de }
\put(41.7,-498.6875){\fontsize{12}{1}\usefont{T1}{cmr}{m}{n}\selectfont\color{color_29791}uma lógica combinacional implementada com portas lógicas AND, OR e NOT. Cada }
\put(41.7,-512.4863){\fontsize{12}{1}\usefont{T1}{cmr}{m}{n}\selectfont\color{color_29791}combinação do Opcode ativa um conjunto específico de sinais, como RegDst, ALUSrc, }
\put(41.7,-526.2852){\fontsize{12}{1}\usefont{T1}{cmr}{m}{n}\selectfont\color{color_29791}MemToReg, RegWrite, Branch, ALUOp0, ALUOp1, Jump, MemWrite, MemRead e Halt. }
\put(41.7,-540.084){\fontsize{12}{1}\usefont{T1}{cmr}{m}{n}\selectfont\color{color_29791}Esses sinais determinam como os dados são manipulados e armazenados ao longo do }
\put(41.7,-553.8828){\fontsize{12}{1}\usefont{T1}{cmr}{m}{n}\selectfont\color{color_29791}ciclo de execução da instrução. Por exemplo, se a instrução for do tipo R, o sinal RegDst }
\put(41.7,-567.6816){\fontsize{12}{1}\usefont{T1}{cmr}{m}{n}\selectfont\color{color_29791}será ativado para indicar que o destino do resultado está no campo rd da instrução. }
\put(41.7,-581.4805){\fontsize{12}{1}\usefont{T1}{cmr}{m}{n}\selectfont\color{color_29791}Instruções que acessam a memória, como lw e sw, ativam MemRead e MemWrite, }
\put(41.7,-595.2793){\fontsize{12}{1}\usefont{T1}{cmr}{m}{n}\selectfont\color{color_29791}respectivamente. O sinal Jump é ativado quando a instrução é um desvio incondicional, }
\put(41.7,-609.0781){\fontsize{12}{1}\usefont{T1}{cmr}{m}{n}\selectfont\color{color_29791}enquanto Branch é ativado para desvios condicionais como beq. A implementação do }
\put(41.7,-622.877){\fontsize{12}{1}\usefont{T1}{cmr}{m}{n}\selectfont\color{color_29791}controlador é baseada em uma tabela verdade que mapeia os valores de Opcode para os }
\put(41.7,-636.6758){\fontsize{12}{1}\usefont{T1}{cmr}{m}{n}\selectfont\color{color_29791}sinais de controle correspondentes. Essa tabela serve como base para a construção do }
\put(41.7,-650.4746){\fontsize{12}{1}\usefont{T1}{cmr}{m}{n}\selectfont\color{color_29791}circuito lógico, garantindo que cada instrução seja corretamente interpretada e executada }
\put(41.7,-664.2734){\fontsize{12}{1}\usefont{T1}{cmr}{m}{n}\selectfont\color{color_29791}pelo processador. Em resumo, o Controlador de Sinais é responsável por coordenar a }
\put(41.7,-678.0723){\fontsize{12}{1}\usefont{T1}{cmr}{m}{n}\selectfont\color{color_29791}operação dos diferentes componentes do processador, garantindo que cada instrução }
\put(41.7,-691.8711){\fontsize{12}{1}\usefont{T1}{cmr}{m}{n}\selectfont\color{color_29791}seja processada corretamente. Ele atua como o "cérebro" da unidade de controle, }
\put(41.7,-705.6699){\fontsize{12}{1}\usefont{T1}{cmr}{m}{n}\selectfont\color{color_29791}traduzindo}
\put(43.20001,-339.1992){\includegraphics[width=441pt,height=291pt]{latexImage_6d8151aac1623033fbe7c2bdd118320c.png}}
\end{picture}
\newpage
\begin{tikzpicture}[overlay]
\path(0pt,0pt);
\filldraw[color_283006][nonzero rule]
(-15pt, 10pt) -- (580.5pt, 10pt)
 -- (580.5pt, 10pt)
 -- (580.5pt, -832.25pt)
 -- (580.5pt, -832.25pt)
 -- (-15pt, -832.25pt) -- cycle
;
\end{tikzpicture}
\begin{picture}(-5,0)(2.5,0)
\put(429.6378,-616.6992){\fontsize{12}{1}\usefont{T1}{cmr}{b}{n}\selectfont\color{color_29791} }
\put(168.4301,-631.562){\fontsize{11}{1}\usefont{T1}{cmr}{b}{n}\selectfont\color{color_29791}Figura 10. Imagem do Controlador do MIPS. }
\put(282.6378,-646.2104){\fontsize{11}{1}\usefont{T1}{cmr}{b}{n}\selectfont\color{color_29791} }
\put(137.1378,-616.6992){\includegraphics[width=291pt,height=568.5pt]{latexImage_f3dcfc2e027d594bcfc6dbeb923a8230.png}}
\end{picture}
\newpage
\begin{tikzpicture}[overlay]
\path(0pt,0pt);
\filldraw[color_283006][nonzero rule]
(-15pt, 10pt) -- (580.5pt, 10pt)
 -- (580.5pt, 10pt)
 -- (580.5pt, -832.25pt)
 -- (580.5pt, -832.25pt)
 -- (-15pt, -832.25pt) -- cycle
;
\end{tikzpicture}
\begin{picture}(-5,0)(2.5,0)
\put(526.95,-171.9492){\fontsize{12}{1}\usefont{T1}{cmr}{b}{n}\selectfont\color{color_29791} }
\put(94.59612,-186.812){\fontsize{11}{1}\usefont{T1}{cmr}{b}{n}\selectfont\color{color_29791}Figura 11. Imagem da Tabela Verdade Utilizada para Criar o Controlador. }
\put(41.7,-212.3984){\fontsize{12}{1}\usefont{T1}{cmr}{m}{n}\selectfont\color{color_29791} }
\put(41.7,-238.1973){\fontsize{12}{1}\usefont{T1}{cmr}{b}{n}\selectfont\color{color_29791}5. CONCLUSÃO }
\put(41.7,-253.9961){\fontsize{12}{1}\usefont{T1}{cmr}{m}{n}\selectfont\color{color_29791} }
\put(41.7,-269.7939){\fontsize{12}{1}\usefont{T1}{cmr}{m}{n}\selectfont\color{color_29791} Este relatório apresentou o desenvolvimento completo de um processador MIPS de }
\put(41.7,-283.5928){\fontsize{12}{1}\usefont{T1}{cmr}{m}{n}\selectfont\color{color_29791}16 bits, abrangendo desde a implementação dos principais componentes até a integração }
\put(41.7,-297.3916){\fontsize{12}{1}\usefont{T1}{cmr}{m}{n}\selectfont\color{color_29791}final do sistema. Foram desenvolvidos e testados módulos essenciais, incluindo o }
\put(41.7,-311.1904){\fontsize{12}{1}\usefont{T1}{cmr}{m}{n}\selectfont\color{color_29791}somador de 6 bits, o contador de programa, a memória de instruções (ROM), o banco de }
\put(41.7,-324.9893){\fontsize{12}{1}\usefont{T1}{cmr}{m}{n}\selectfont\color{color_29791}registradores, a Unidade Lógica e Aritmética (ULA), a memória de dados (RAM), o }
\put(41.7,-338.7881){\fontsize{12}{1}\usefont{T1}{cmr}{m}{n}\selectfont\color{color_29791}extensor de bits e o controlador de sinais. Além disso, o conjunto de instruções foi }
\put(41.7,-352.5869){\fontsize{12}{1}\usefont{T1}{cmr}{m}{n}\selectfont\color{color_29791}totalmente definido e implementado, garantindo a correta execução de operações }
\put(41.7,-366.3857){\fontsize{12}{1}\usefont{T1}{cmr}{m}{n}\selectfont\color{color_29791}aritméticas, lógicas, de memória e controle de fluxo. A validação dos circuitos no Logisim }
\put(41.7,-380.1846){\fontsize{12}{1}\usefont{T1}{cmr}{m}{n}\selectfont\color{color_29791}permitiu verificar a operação integrada do processador, confirmando que os módulos }
\put(41.7,-393.9834){\fontsize{12}{1}\usefont{T1}{cmr}{m}{n}\selectfont\color{color_29791}funcionam de maneira coordenada e eficiente. O controlador de sinais e a ULA foram }
\put(41.7,-407.7822){\fontsize{12}{1}\usefont{T1}{cmr}{m}{n}\selectfont\color{color_29791}fundamentais para garantir a correta interpretação das instruções, possibilitando a }
\put(41.7,-421.5811){\fontsize{12}{1}\usefont{T1}{cmr}{m}{n}\selectfont\color{color_29791}execução de programas completos no sistema desenvolvido. Com a finalização deste }
\put(41.7,-435.3799){\fontsize{12}{1}\usefont{T1}{cmr}{m}{n}\selectfont\color{color_29791}projeto, o processador MIPS de 16 bits agora possui todas as funcionalidades essenciais }
\put(41.7,-449.1787){\fontsize{12}{1}\usefont{T1}{cmr}{m}{n}\selectfont\color{color_29791}para processar instruções de maneira autônoma e eficiente. A abordagem modular }
\put(41.7,-462.9775){\fontsize{12}{1}\usefont{T1}{cmr}{m}{n}\selectfont\color{color_29791}adotada demonstrou-se eficaz, permitindo uma construção organizada e facilitando }
\put(41.7,-476.7764){\fontsize{12}{1}\usefont{T1}{cmr}{m}{n}\selectfont\color{color_29791}futuras expansões. O desenvolvimento desse processador proporcionou um entendimento }
\put(41.7,-490.5752){\fontsize{12}{1}\usefont{T1}{cmr}{m}{n}\selectfont\color{color_29791}aprofundado dos princípios da arquitetura MIPS, reforçando a importância do design }
\put(41.7,-504.374){\fontsize{12}{1}\usefont{T1}{cmr}{m}{n}\selectfont\color{color_29791}lógico e da integração de hardware na construção de sistemas computacionais. }
\put(41.7,-520.1738){\fontsize{12}{1}\usefont{T1}{cmr}{m}{n}\selectfont\color{color_29791} }
\put(41.7,-535.9727){\fontsize{12}{1}\usefont{T1}{cmr}{m}{n}\selectfont\color{color_29791} }
\put(41.7,-551.7705){\fontsize{12}{1}\usefont{T1}{cmr}{b}{n}\selectfont\color{color_29791}6. REFERÊNCIAS }
\put(41.7,-567.5703){\fontsize{12}{1}\usefont{T1}{cmr}{m}{n}\selectfont\color{color_29791} }
\put(41.7,-583.3691){\fontsize{12}{1}\usefont{T1}{cmr}{m}{n}\selectfont\color{color_29791}OLIVEIRA, Ruy de. Microprocessadores MIPS. Material didático. Disponibilizado em aula }
\put(41.7,-597.168){\fontsize{12}{1}\usefont{T1}{cmr}{m}{n}\selectfont\color{color_29791}pelo professor. Acesso em: 11 Fev. 2025. }
\put(41.7,-612.9658){\fontsize{12}{1}\usefont{T1}{cmr}{m}{n}\selectfont\color{color_29791}TANENBAUM, Andrew S.; AUSTIN, Todd M. Organização Estruturada de Computadores. }
\put(41.7,-626.7646){\fontsize{12}{1}\usefont{T1}{cmr}{m}{n}\selectfont\color{color_29791}6. ed. São Paulo: Pearson, 2014. }
\put(41.7,-642.5645){\fontsize{12}{1}\usefont{T1}{cmr}{m}{n}\selectfont\color{color_29791}PATTERSON, David A.; HENNESSY, John L. Computer Organization and Design: The }
\put(41.7,-656.3633){\fontsize{12}{1}\usefont{T1}{cmr}{m}{n}\selectfont\color{color_29791}Hardware/Software Interface. 5th ed. Amsterdam: Morgan Kaufmann, 2014. }
\put(41.7,-672.1621){\fontsize{12}{1}\usefont{T1}{cmr}{m}{n}\selectfont\color{color_29791}HENNESSY, John L.; PATTERSON, David A. Computer Organization and Design: The }
\put(41.7,-685.9609){\fontsize{12}{1}\usefont{T1}{cmr}{m}{n}\selectfont\color{color_29791}Hardware/Software Interface. 5. ed. Amsterdam: Elsevier, 2014. }
\put(41.7,-701.7588){\fontsize{12}{1}\usefont{T1}{cmr}{m}{n}\selectfont\color{color_29791}LOGISIM. Logisim Evolution: Digital Logic Simulator. Disponível em: }
\put(41.7,-715.5576){\fontsize{12}{1}\usefont{T1}{cmr}{m}{n}\selectfont\color{color_29791}https://logisim-evolution.org.}
\put(43.20001,-171.9492){\includegraphics[width=482.25pt,height=123.75pt]{latexImage_3eb5e57e176f3945b7ed5e17f1ccdf3c.png}}
\end{picture}
\end{document}